\documentclass[a4paper,12pt]{article}

\usepackage[english]{babel}
\usepackage{courier}


\usepackage[T1]{fontenc}

\usepackage{listings}
\lstloadlanguages{python}
\title{ILC DIRAC, a grid solution for the LC community}
\author{S.~Poss}

\begin{document}
\maketitle
\abstract{This document presents the different parts of the ILCDIRAC framework.
It is not intended as a manual, but as a support for the maintainer. It shows
the different elements, and details the corresponding code. It is clear that
this document will also show usage of the different bits, but not thoroughly.}

\section{Setups}
ILCDIRAC makes use of 2 setups:
\begin{itemize}
  \item ILC-Production: Main setup, production and user activity uses that
  setup. Running on volcd01.cern.ch (most services/agents, web server),
  volcd02.cern.ch (FileCatalogDB (not the service), overlay system), volcd03.cern.ch (LogSE)
  \item ILC-Developement: Used for testing new functionnality and new DIRAC
  releases. Runs on volcd03.cern.ch
\end{itemize}

\section{General structure}
The ILCDIRAC framework has the following structure:
\begin{itemize}
  \item Core: ILCDIRAC utilities used throughout the code
  \begin{itemize}
    \item script: ILCDIRAC specific scripts, detailed later
    \item Utilitites: Set of utilities used in the Workflow modules, and
    detailed later
  \end{itemize}
  \item Interfaces: Interface to DIRAC (and ILCDIRAC by extension)
  \begin{itemize}
    \item API: as it's name suggests
    \begin{itemize}
      \item NewInterface
      \begin{itemize}
        \item Examples: how to use the new interface
        \item Code for the new interface, detailed later
      \end{itemize}
      \item Examples
      \item and the old API code not maintained, but kept for backward
      compatibilty, except the \emph{DiracILC} code, still to be used. Detailed
      later
    \end{itemize}
    \item scripts: Set of scripts using the ILCDIRAC interface, detailed later.
    Some are obsolete or not maintained. Details follow.
  \end{itemize}
  \item OverlaySystem: Control the behavior of the overlay jobs, prevents
  killing the SRM
  \begin{itemize}
    \item Agent: Agent running on volcd03, resets the counters per site. Details
    below
    \item Client: Client to connect to the Overlay Service, exposes the
    functionnality of the server
    \item DB: Database definition. Schema is shown later
    \item Service: Service runnign on volcd03: essentially stores how many jobs
    are downloading the overlay files at a given site, and prevents a job from
    running in case there are too many. Code details are below.
  \end{itemize}
  \item ProcessProductionSystem: As of \today\ still being developed. Aim
  is to have a service that handles the production to reduce the amount of human work:
  deploy applications, remove them, store relation between software and
  production, between data and productions, production details\ldots
  \begin{itemize}
    \item Agent: Agents running on one of the VO boxes (only in dev setup for
    the moment)
    \begin{itemize}
      \item DataRecoveryAgent: Recover failed jobs that did not report the File
      status, typically when pilots are killed.
      \item ProductionSummaryAgent: Collect the statistics for each production,
      produce nicely formatted web page containing also the production details
      \item SoftwareManagementAgent: Install/remove software from all sites,
      update availability
    \end{itemize}
    \item Client: Client for ProcessProductionHandler (service)
    \item DB: Database holding all the info for this service, detailed later
    \item Service: As name suggested serves the DB mostly.
    \item Utilities: Specific module for software management, details later
  \end{itemize}
  \item SoftwareManagement: Obsolete, as the functionnality was moved to
  ProcessProductionSystem, kept here for completeness. Not detailed
  \item Workflow: What runs on the grid
  \begin{itemize}
    \item Modules: they are completely detailed later
  \end{itemize}
\end{itemize}


\section{Interface}

\section{Workflow Modules}

\section{Core Utilities}

\section{Overlay System}

\section{Process Production System}

\section{Making releases}

\end{document}