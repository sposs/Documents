\documentclass[a4paper,12pt]{article}
\usepackage{tikz}
\usetikzlibrary{arrows,shapes}

\usepackage[english]{babel}
\usepackage{courier}


\usepackage[T1]{fontenc}

\usepackage[pdftex, colorlinks=true]{hyperref}

\usepackage{listings}
\lstloadlanguages{python}
\lstset{language=python,numbers=left,numberstyle=\tiny,emph={self},emphstyle=\color{blue},basicstyle=\footnotesize}
\title{ILC DIRAC, a grid solution for the LC community}
\author{S.~Poss}

\begin{document}
\tikzstyle{decision} = [diamond, draw, text badly centered, node distance=2.8cm]
\tikzstyle{block} = [rectangle, draw, text centered, rounded corners, node distance=2.2cm]
\tikzstyle{autoblock} = [rectangle, draw, text centered, rounded corners, node distance=2.2cm]
\tikzstyle{line} = [draw, -triangle 90]
\tikzstyle{dline} = [draw, dashed, -triangle 90]

\maketitle
\abstract{This document presents the different parts of the ILCDIRAC framework.
It is not intended as a manual, but as a support for the maintainer. It shows
the different elements, and details the corresponding code. It is clear that
this document will also show usage of the different bits, but not thoroughly.}

\tableofcontents

\section{Setups}
ILCDIRAC makes use of 2 setups:
\begin{itemize}
  \item ILC-Production: Main setup, production and user activity uses that
  setup. Running on volcd01.cern.ch (most services/agents, web server),
  volcd02.cern.ch (FileCatalogDB (not the service), overlay system), volcd03.cern.ch (LogSE)
  \item ILC-Developement: Used for testing new functionnality and new DIRAC
  releases. Runs on volcd03.cern.ch
\end{itemize}

\section{General structure}
The ILCDIRAC framework has the following structure:
\begin{itemize}
  \item Core: ILCDIRAC utilities used throughout the code, Sec.~\ref{core}
  \begin{itemize}
    \item script: ILCDIRAC specific scripts, detailed later
    \item Utilitites: Set of utilities used in the Workflow modules, and
    detailed later, Sec.~\ref{coreutilities}
  \end{itemize}
  \item Interfaces: Interface to DIRAC (and ILCDIRAC by
  extension) Sec.~\ref{interface}
  \begin{itemize}
    \item API: as it's name suggests
    \begin{itemize}
      \item NewInterface
      \begin{itemize}
        \item Examples: how to use the new interface
        \item Code for the new interface, detailed later, Sec.~\ref{application}
        and Sec.~\ref{job}
      \end{itemize}
      \item Examples
      \item and the old API code not maintained, but kept for backward
      compatibilty, except the \emph{DiracILC} code, still to be used.
      Sec.~\ref{diracilc}
    \end{itemize}
    \item scripts: Set of scripts using the ILCDIRAC interface, detailed later.
    Some are obsolete or not maintained. Sec.~\ref{interfacescripts}
  \end{itemize}
  \item OverlaySystem: Control the behavior of the overlay jobs, prevents
  killing the SRM. Sec.~\ref{overlaysys}
  \begin{itemize}
    \item Agent: Agent running on volcd03, resets the counters per site. Details
    below
    \item Client: Client to connect to the Overlay Service, exposes the
    functionnality of the server
    \item DB: Database definition. Schema is shown later
    \item Service: Service runnign on volcd03: essentially stores how many jobs
    are downloading the overlay files at a given site, and prevents a job from
    running in case there are too many. Code details are below.
  \end{itemize}
  \item ProcessProductionSystem (Sec.~\ref{processprodsys}): As of \today\
  still being developed. Aim is to have a service that handles the production to reduce the amount of human work:
  deploy applications, remove them, store relation between software and
  production, between data and productions, production details\ldots
  \begin{itemize}
    \item Agent: Agents running on one of the VO boxes (only in dev setup for
    the moment)
    \begin{itemize}
      \item DataRecoveryAgent: Recover failed jobs that did not report the File
      status, typically when pilots are killed.
      \item ProductionSummaryAgent: Collect the statistics for each production,
      produce nicely formatted web page containing also the production details
      \item SoftwareManagementAgent: Install/remove software from all sites,
      update availability
    \end{itemize}
    \item Client: Client for ProcessProductionHandler (service)
    \item DB: Database holding all the info for this service, detailed later
    \item Service: As name suggested serves the DB mostly.
    \item Utilities: Specific module for software management, details later
  \end{itemize}
  \item SoftwareManagement: Obsolete, as the functionnality was moved to
  ProcessProductionSystem, kept here for completeness. Not detailed
  \item Workflow: What runs on the grid
  \begin{itemize}
    \item Modules: they are completely detailed later, Sec.~\ref{modules}
  \end{itemize}
\end{itemize}


\section{Interface}\label{interface}
Here we detail the content of every class in this package. The Examples are
treated separately, as theyu only use the functionnality exposed here. The order
is as follows: the job and application classes are presented then the DiracILC
class. The old API is not presented as not maintained, and will be dropped in one of the future
releases.

\subsection{General considerations}
When reviewing the Interface, the motivation was to render ther whole system
easier to use and to maintain. The first idea is to separate the job and the
applications that should run in the job. The reason is that it allows for a
better flexibility, namely a user can predefine a set of applications,
independantly of the job they run in.

In Fig.~\ref{fig:defdefinejob} the usual process of defining a job is shown. 
\begin{figure}[h]
\begin{center}
\begin{tikzpicture}[scale=0.8,auto]
\matrix [column sep=7mm, row sep=4mm,ampersand replacement=\&]
{
\node [block] (app1) {Define Application 1};\\
\node [block] (app2) {Define Application 2};\\
\node [block] (job) {Define Job};\\
\node [block] (submit) {Submit job};\\
};
\path [line] (app1) -- (app2);
\path [line] (app2) -- (job);
\path [line] (job) -- (submit);
\end{tikzpicture}
\end{center}
\caption{Usual process of job definition}
\label{fig:defdefinejob}
\end{figure}
In the ILCDIRAC framework, this process is modified to separate the applications
and the job, as shown in Fig.~\ref{fig:definejob}.
\begin{figure}[h]
\begin{center}
\begin{tikzpicture}[scale=0.8,auto]
\matrix [column sep=7mm, row sep=4mm,ampersand replacement=\&]
{
\node {App definition};\&
\node {Job definition};\\
\node [block] (app1) {Application 1};\&
\node [block] (job) {Define Job};\\

\node [block] (app2) {Application 2};\&
~\\
~\&
\node [block] (submit) {Submit job};\\
};
\path [line] (app1) -- (job);
\path [line] (app2) -- (job);
\path [line] (job) -- (submit);
\end{tikzpicture}
\end{center}
\caption{ILCDIRAC process of job definition. The arrows between application and
job stand for job.append(application). This is
technically explained later.}
\label{fig:definejob}
\end{figure}

The applications are ran on the GRID using the DIRAC workflow system. This is
briefly explained in Sec.~\ref{workflowsys}.

\subsection{Application}\label{application}
This class is the base application class, all applications must inherit from
this. 

The general principle in the application class comes from the fact that all
applications that runs on the grid share a set of parameters. Those are:
\begin{itemize}
  \item \emph{Application name}: Name of the application, e.g. marlin, SLIC,
  lcio, etc.
  \item \emph{Application version}: As most software uses versionning with e.g.
  SVN, there are different versions that correspond to a set of functionnality. When
  willing to run several versions of the software in parallel, this is needed.
  \item \emph{Options (or steering) file}: All application need to be told what
  they need to do. 
  \item \emph{Input file (or none)}: Most application used in HEP are designed
  to transform a set of data into another. Physics generators are an exception as
  they only produce files.
  \item \emph{Output file}: All applications produce something: binary data,
  text files, logs, etc. Those should be handled.
  \item \emph{Log file}: To debug, it's needed to keep the stdout of the
  application.
\end{itemize}
As this was designed in the context of ILCDIRAC, there are several other
parameters that are shared between the applications:
\begin{itemize}
  \item \emph{Number of events}: The number of events that will be ran by the
  application. Depending on the application, this can be omitted.
  \item \emph{Energy}: The sample's energy. Can be needed to determine which
  calibration to use, or, in ILC's case, the overlay files to use.
\end{itemize}
There are other common functionnality that will be detailed later as they are
use-specific.

\begin{lstlisting}[name=appli]
from DIRAC.Core.Workflow.Module         import ModuleDefinition
from DIRAC.Core.Workflow.Parameter      import Parameter

from DIRAC import S_OK,S_ERROR, gLogger
import inspect, sys, string, types, os
\end{lstlisting}

The first import statement brings the necessary utilities to build a Module. The
second is taking care of the Parameter. Both functionalities are related to the
Workflow mechanism, not discussed here. Without detailing, a module is the base
element of a workflow, while a parameter is a class holding a parameter (who
would have guessed?). A Parameter is attached to a Module (like in the
Application case), to a Step (like in the Job, see later), or to the Workflow
(see the Job also). 

We also import the usual DIRAC utilities to handle returned values, as well as a
logger utility.

The last imports are used to check the method calls: the arguments must have a
specific type, checked. Those check methods are detailed below.

\begin{lstlisting}[name=appli]
class Application(object):
  """ General application definition. Any new application 
  should inherit from this class.
  """
\end{lstlisting}
The application class does not depend on anything else than the base ``object''
class, from which all python class should inherit.

\subsubsection{The constructor}
\begin{lstlisting}[name=appli]
  def __init__(self, paramdict = None):
\end{lstlisting}
The constructor takes a dictionary of the parameters of the class. This allows a
one-line definition of the application. This functionnality was designed to
allow easy backward compatibility.

~

Most documentation on the content of the constructor can be found in the code
itself, and thus is not repeated here. One important point to notice: all members that 
should not be printed when checking the job (see later)
are private members and therefore start with \_, e.g. 
\begin{lstlisting}[firstnumber=53,belowskip=0.5cm]
    self._listofoutput
\end{lstlisting}

~

The 
\begin{lstlisting}[firstnumber=71]
    self.accountInProduction = True    
\end{lstlisting}
is necessary to have the application considered in the production jobs, in
particular in the output data accounting. An exemple for an application where
this would be set to false is when the same application is to run twice in a
given job, and only the output of the second call should be stored. A specific
setter is designed, see later.

~

The most important part of this constructor consists of the following lines:
\begin{lstlisting}[firstnumber=73]
    self._modulename = ''
    self._moduledescription = ''
    self._importLocation = "ILCDIRAC.Workflow.Modules" 
\end{lstlisting}
They are set by each application's constructor and are used by the workflow
framework to identify the location of the python module to be executed, as well
as the description of such module. 

~

The following lines
\begin{lstlisting}[firstnumber=81]
    self._jobapps = []
    self._jobsteps = []
    self._inputapp = []
    self._linkedidx = None
    #Needed to link the parameters.
    self._inputappstep = None
\end{lstlisting}
are essential too: they allow the current application to know all about the job.
This is needed when linking applications together (output of one is to be used
as input to the next). The linking is described later.

~

The line
\begin{lstlisting}[firstnumber=93]
    self._log = gLogger.getSubLogger(self.__class__.__name__)
\end{lstlisting}
sets the logger: the application class name is used.

~

Finally, the dictionary passed to the application is used in
\begin{lstlisting}[firstnumber=97]
    self._setparams(paramdict)
\end{lstlisting}
where the different parameter values are assigned. 

\subsubsection{The methods}
The following method
\begin{lstlisting}[firstnumber=99]
  def __repr__(self):
    str  = "%s"%self.appname
    if self.version:
      str += " %s"%self.version
    return str
\end{lstlisting}
is useful to allow easy dump of the applciation: it allows for 
"\lstinline[language=python]!print application!", 
which show on the stdout the application name and version (if defined).

~

The method
\begin{lstlisting}[firstnumber=105]
  def _setparams(self,params):
    if not params:
      return S_OK()
    for param,value in params.items():
      if type(value) in types.StringTypes:
        value = "'%s'"%value
      try:
        exec "self.set%s(%s)"%(param,str(value))
      except:
        self._log.error("The %s class does not have 
                         a set%s method."%(self.__class__.__name__,
                                           param))
    return S_OK()  
\end{lstlisting}
is called by the constructor takes as argument the dictionnary that was passed
in the init. This method then calls the setters using the folowing: if an
application has a setter \lstinline[language=python]!setX(Y)!, 
then the dictionary to pass to the constructor must be
\lstinline[language=python]!{"X":Y}!.
The method checks that the class has the proper setter and returns an error if
not.

~

The following methods are the setters for the common functionality. Nothing is
particular about those, so only the def statement is shown.
\begin{lstlisting}[firstnumber=118,stepnumber=118,numberfirstline=true]
  def setName(self,name):
  def setVersion(self,version):
  def setSteeringFile(self,steeringfile):
  def setLogFile(self,logfile):
  def setNbEvts(self,nbevts):
  def setNumberOfEvents(self,nbevts):
  def setEnergy(self,energy):
\end{lstlisting}
Notice that each method has a call to \lstinline!self._checkArgs! that allows
for proper type checking of the input values.

~

The following setters are related to the definition of the output data.
\begin{lstlisting}[firstnumber=187]
  def setOutputFile(self,ofile, path = None):
    self._checkArgs({ 'ofile' : types.StringTypes } )
    
    self.outputFile = ofile
    self.prodparameters[ofile]={}
    if self.detectortype:
      self.prodparameters[ofile]['detectortype'] = self.detectortype
    if self.datatype:
      self.prodparameters[ofile]['datatype']= self.datatype
    
    if path:
      self._checkArgs({ 'path' : types.StringTypes } )
      self.outputPath = path
      
    return S_OK()  
\end{lstlisting}
The most striking feature of this method is the fact that there is a dictionary,
\lstinline!self.prodparameters! that is filled with some info. This dictionary
is used in the ProductionJob class.

It is also needed to define the Outut Storage Element (SE) for the production
jobs, and that's done with:
\begin{lstlisting}[firstnumber=210]  
  def setOutputSE(self,se):
    self._checkArgs({ 'se' : types.StringTypes } )
    self.outputSE = se
    return S_OK()
\end{lstlisting}

~

Next, the method for the definition of the input data to the application is
show:
\begin{lstlisting}[firstnumber=221]  
  def setInputFile(self,inputfile):
    kwargs = {"inputfile":inputfile}
    if not type(inputfile) in types.StringTypes and 
       not type(inputfile)==type([]):
      return self._reportError("InputFile must be string or list of strings",
                               __name__,**kwargs)
    if not type(inputfile)==type([]):
      inputfile = [inputfile]
    for f in inputfile:
      if os.path.exists(f) or f.lower().count("lfn:"):
        self.inputSB.append(f)
        
    self.inputfile = string.join(inputfile,";")

    return S_OK()
\end{lstlisting}
The relevant part of this method is the fact that if the file specified as
argument exists locally or starts with ``lfn:'' then it's appended to the
application input sandbox. This sandbox will be added to the Job's sandbox when
needed.

~

The next method:
\begin{lstlisting}[firstnumber=240]  
  def setForgetAboutInput(self):
    
    self.forget_about_Input = True
    
    return S_OK()
\end{lstlisting}
allows the application to ignore the handling of the input file. To understand
the motivation of such method, it's necessary to get back to the basics.
Usually, a user will test locally his/her steering (option) file and will then
want to use it on a mass production system to process a much larger data set. A
user usually wants to be able to do that without having to manually change the
steering file for every job. For that, the modules that run the applications in
ILCDIRAC handle the input data of the job, and change the steering file
accordingly. This is not mandatory, and depends on the use case. But sometimes,
such override of the input data is not needed. An example is that the job uses
direct access to the disk on the job's running site, and therefore no input
data resolution is needed. Then the job should use whatever is specified in the
steering without changing the value. And it should also not complain about
missing input. This method was designed for that purpose.

~

\begin{lstlisting}[firstnumber=248]  
  def getInputFromApp(self,application):
    """ Called to link applications
    
    >>> app1 = App1()
    >>> app2 = App2()
    >>> app2.getInputFromApp(app1)
    
    @param application: Application to link against.
    @type application: application
    """
    self._inputapp.append(application)
    return S_OK()  
\end{lstlisting}
is necessary for the linking of applications: when the second application is
supposed to be able to take it's input from the output of the first.

~

\begin{lstlisting}[firstnumber=261]  
  def setDebug(self,debug = True):
    """ Set the application to debug mode
    
    >>> app = Application()
    >>> app.setDebug()
    
    @param debug: Set the application to debug mode. 
      Default is True when called. If not, then it's false.
    @type debug: bool
    """
    self._checkArgs({ "debug": types.BooleanType} )
    self.debug = debug
    return S_OK()
\end{lstlisting}
sets a debug flag for the application: allows for on-site verbosity
adjustements.

~

Finally, the method 
\begin{lstlisting}[firstnumber=274]  
  def listAttributes(self):
    """ Method to list attributes for users. Doesn't list any 
    private or semi-private attributes
    """
    print 'Attribute list :'
    for key,val in self.__dict__.items():
      if not key[0]=="_":
        if not val:
          val = "Not defined"
        print "  ",key,":",val
\end{lstlisting}
is used when checkling the job. It's called by the Job class, and lists the
public members' values, for debugging before submission. More on that in the Job
class description, in Sec.~\ref{job}.

\subsection{GenericApplication}\label{genericapp}
This application is intended to run a script based on an application. 

\subsection{GetSRMFile}\label{getsrmfile}
When a file is not in the File Catalog, it can still be access using this
application.

\subsection{RootScript}\label{rootscript}
Run a ROOT based script, shell or python.

\subsection{RootMacro}\label{rootmacro}
Run a ROOT macro (.C).

\subsection{Whizard}\label{whizard}
Generate events using Whizard.

\subsection{Pythia}\label{pythia}
Generate events using PYTHIA standalone.

\subsection{PostGenSelection}\label{postgensel}
This utility was created for JJ Blaising, but not used anymore.

\subsection{StdhepCut}\label{stdhepcut}
Used to apply generator level cuts on the final stdhep files. Application code
written by Lars Weuste, extended by John Marshall and Astrid Munnich. 

\subsection{StdHepSplit}\label{stdhepsplit}
Cut a stdhep file in chunks. Mostly used for the SID production

\subsection{Mokka}\label{mokka}
Run Mokka: ILD simulation

\subsection{SLIC}\label{slic}
Run SLIC: SiD simulation

\subsection{OverlayInput}\label{overlay}
Download the files for the overlay.

\subsection{Marlin}\label{marlin}
Run Marlin: reconstruction for ILD

\subsection{LCSIM}\label{lcsim}
Run LCSIM: Reconstruction for SiD

\subsection{SLICPandora}\label{slicpan}
Apply Pandora PFO on SiD reconstructed data

\subsection{CheckCollections}\label{checkcoll}
Make sure a set of collection is available in the input file.

\subsection{SLCIOConcatenate}\label{slcioconcat}
Concatenate many slcio files into one. Useful for the ``Merge'' jobs. 

\subsection{SLCIOSplit}\label{slciosplit}
Split a slcio file into many chunks. 

\subsection{Tomato}\label{tomato}
Run the Tomato analysis on an input file.

\subsection{Job}\label{job}
This class is the base Job class, all job types must inherit from this class.

\subsubsection{UserJob}\label{userjob}
User job class for user jobs\ldots

\subsubsection{ProductionJob}\label{prodjob}
Production job class. The main different with the previous job class is the fact
that it's not intended to be submitted to Dirac, but only to the
TransformationSysten.

\subsubsection{SIDProductionJob}\label{sidprod}
SIDProductionJob. Same as the previous class, but implementing SiD production
specific functionnality.

\subsection{DiracILC}\label{diracilc}
This class is the interface between the Job and DIRAC.

\subsection{scripts}\label{interfacescripts}
Several scripts are developed to provide specific funtionnality for ILCDIRAC.

\section{Workflow Modules}\label{modules}
The following are the modules running on the grid, executing the different
applications.

\section{Core}\label{core}
Core functionnality.
\subsection{Utilities}\label{coreutilities}
The following utilities are used to provide specific functions.
\subsection{scripts}\label{corescripts}
The scripts\ldots

\section{Overlay System}\label{overlaysys}
Make sure the jobs don't kill the disks.

\section{Process Production System}\label{processprodsys}
Big service\ldots

\section{Making releases}\label{makingrelease}
Here we detail the manner to create a release.

\section{Workflow mechanism}\label{workflowsys}
\end{document}