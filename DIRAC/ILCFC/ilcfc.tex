\documentclass[a4paper,12pt]{article}
\usepackage{tikz}
\usetikzlibrary{arrows,shapes}

\usepackage{courier}

\usepackage[T1]{fontenc}

\usepackage[pdftex, colorlinks=true]{hyperref}

\usepackage[british]{babel}

\usepackage{listings}
\lstloadlanguages{python}
\lstset{language=python,emph={self},emphstyle=\color{blue},basicstyle=\footnotesize}
\title{FileCatalog structure and Content for the CLIC CDR}
\author{S.~Poss}

\begin{document}

\tikzstyle{decision} = [diamond, draw, text badly centered, node distance=2.8cm]
\tikzstyle{block} = [rectangle, draw, text centered, rounded corners, node distance=2.2cm]
\tikzstyle{autoblock} = [rectangle, draw, text centered, rounded corners, node distance=2.2cm]
\tikzstyle{line} = [draw, -triangle 90]
\tikzstyle{dline} = [draw, dashed, -triangle 90]

\maketitle
\abstract{This document presents how the File Catalog was organized for the
CLIC CDR. It also gives the searchable meta data information currently
available.}

\tableofcontents

\section{Introduction}
The CLIC CDR mass production used the DIRAC File Catalog (FC) as replica and
meta data catalog. This proved to be an efficient solution, and made bookkeeping
straight forward for the data manager. 

This File Catalog allows for storing the
files in different directories, much like any file system. The Logical File
Names (LFN) allows to obtain the Physical File Names (PFN) by adding the server
details (srm end point and base path) of one Storage Element containing the
file. 

As for the meta data, there are two kinds: searchable and non-searchable. Meta
data can be attached to directories and/or to files, both for searchable and
non-searchable. Obviously, searchable meta data at the file level is much less
efficient than at the directory level, therefore non was used for the CLIC CDR.
Non searchable meta data at the file level was added at a late stage, but holds
some information related to the file.

The File Catalog also hold ancestor-daughter relationships between the files,
allowing to navigate easily from one step to the next.

In the following sections, we first show the chosen path and file name
conventions used, and then we discuss the available searchable meta data.

\section{Organisation}
This section describes how the path structure of the FC was set up. 

The base element is by convention the VO, in this case ilc. This allows to have
multiple VOs using the same File Catalog. At CERN, some people from the CALICE
collaboration are also using DIRAC, and the same setup, and their data appears
under /calice.

The second level distinguishes ``production'' and ``user'' activities, so their
paths are the following:
\begin{itemize}
  \item /ilc/prod for production,
  \item /ilc/user for user data.
\end{itemize} 
In the following, only the production directories are relevant. 

As two machines are running under the same VO, the next level is the machine,
ilc or clic.  In the case of the ilc directory, the data structure available
follows what was produced by Mikael and Akiya for the generation. We will
discuss this at a later stage.

For the clic structure, the sub directory are organized like this:
Energy/EvtType/gen/ProdID/000 for generator level, or
Energy/EvtType/DetectorType/Datatype/ProdID/000
\begin{itemize}
  \item Energy: energy of the machine: 3tev, 1.4tev, 500gev, or 350gev
  \item EvtType: the event type, e.g. ``WW''
  \item DetectorType: as the mane suggests, either ``ILD'' or ``SID''
  \item Datatype: either ``SIM'', ``REC'', or ``DST''
  \item ProdID: the internal production ID of DIRAC, as we are using its
  Transformation System. It has leading 0 such that the length of the string is
  8 chars long.
  \item The last 000 is the taskID\%1000: most Storage Elements and the File
  Catalog do not like having too many files in the same directory, for
  optimization reasons. So those sub directories contain at most 1000 files.
\end{itemize}

So the full path structure is then the following e.g.:\\
/ilc/prod/clic/1.4tev/ch1ch1/gen/00001186/000/ for generator level data, or \\
/ilc/prod/clic/1.4tev/ch1ch1/SID/REC/00001189/000/ for REC files produced with
the SID detector.

~\\

The file names also follow some conventions: The are built using the minimal set
of information needed to obtain their details: event type, data type, production
ID, task ID in that production, and, in case of splitting, a sub file index. For example,
in the files produced in the generator step for the sample above, ch1ch1, the
file names are e.g.: ch1ch1\_gen\_1186\_96.stdhep. From this file name, one
can go to the Production Monitoring page, find the production 1186, check it's
properties, look at the corresponding jobs, and find the task 96 (not the same
as the job ID) by looking at the job names\footnote{Here the job name is
00001186\_00000096.}, and obtain the job's details. In some future, log files
for a particular job should also be made available from the web portal, as the
functionality is available.

In case of splitting at the lcio level, the file name is e.g.\\
ch1ch1\_sim\_1188\_98-3-200.slcio. The last part gives the file index in the
splitting, and the number of events in the file. This is a convention
coming from the ``lcio split'' utility. 

In case of stdhep split, the convention is different, but we have no files in
the FC yet produced with the stdhep splitter.

\section{Meta data}
The meta data is separated in two catagories: searchable and non searchable.
They are presentesd in the following sections.

\subsection{Searchable meta data}
The aim of searchable meta data is to obtain efficiently a data set. This data
set is to be used either as input to productions or for a user to run analysis.
In both case, a hierarchisation of the information is needed. In practice, it
was chosen to use mostly the same information as the path structure. Indeed, the
searchable meta data used is attached to directories. Below is the list of meta
data keys that can be used for setting and reading:
\begin{itemize}
  \item EvtType (string): event type, same as in path
  \item Datatype (string): data type: gen, SIM, REC, DST, same as in path
  \item Energy (string): same as in path
  \item DetectorType (string): same as in path
  \item Machine (string): same as in path (clic or ilc)
  \item ProdID (int): same as in path, but the leading 0 removed
  \item DetectorModel (string): not used yet, but can hold e.g. sidloi3 
  \item MachineParams (string): not used, but meant to hold ilc machine
  parameters name (b1\_ws for e.g.)
  \item NumberOfEvents (int): common number of events in each file in the sub
  directories. This is used to define some type of productions (Mokka), but is
  not used in said production, as this number is now stored at the file level
  (see below).
  \item Luminosity (int): same as NumberOfEvents, used to obtain the
  corresponding cross section before is was stored for every file.
  \item Owner (string): only used for users' files, under /ilc/user
  \item JobType (string): not used, was meant to store the job type
  (generation, simulation, reconstruction, etc.) but overlaps with the other
  metadata
  \item runnumber (int): not used and not needed (legacy from initial
  development)
  \item BXoverlayed (float): not used, was meant to be filled when having
  overlay, but the overlay info can be obtained from the Transformation system. 
  \item StartDate (date): not used, but is meant to hold the start date of the
  production. We do not need it.
\end{itemize}
Using this meta data is described in section~\ref{sec:usingfc}.

\subsection{Non searchable meta data}
This feature was added at a late stage compared to the CLIC CDR mass production
time scale. Therefore, most data does not use this type of meta data. But the
data produced for the volume 3 makes use of it. 

Any string meta data can be used, there is simply a restriction in the length of
the string. File annotations are among the future improvements and they will
not have size limitations.

As this is non searchable, any key can be attached to any directory/file. As a
convention, we use the following at directory level:
\begin{itemize}
  \item SWPackage: software packages used for a given production, e.g.
  whizard.SUSY\_V22, or \\lcsim.CLIC\_CDR;SLICPandora.V6;lcsim.CLIC\_CDR
\end{itemize}

~\\

At the file level, more meta data is available:
\begin{itemize}
  \item Luminosity: for whizard files only
  \item NumberOfEvents: for any file
  \item AdditionalInfo: any file in principle, we set some only
  for files that are produced by Whizard, and is by convention a
  python dictionary: \\ \{'xsection': \{'sum': \{'err\_xsection':
  0.20100000000000001, \\'xsection': 213.67169000000001, 'fraction': 100.0\}\}\}
\end{itemize}
Anything can be added, but because of the string length limitation, only 255
chars are allowed (important for the last field).

~\\

Additionaly, the ancestor-daughter relationship is stored allowing to obtain the
full history of a given file. Using this feature will be described in the next
section.

\section{Interacting with the File Catalog}\label{sec:usingfc}
There are several ways to interact with the file catalog: the CLI, the API, and
in a near future, a web client.

\subsection{The CLI}
The command line interface (CLI) can be invoked by typing in a terminal
\emph{dirac-dms-filecatalog-cli}. This gives a shell like interface, where
commands are to be entered, for example:
\begin{lstlisting}
FC:/> help

Documented commands (type help <topic>):
========================================
add         chgrp      exit  guid meta     replicas  rmreplica 
ancestor    chmod      find  id   mkdir    replicate size      
ancestorset chown      get   lcd  pwd      rm        unregister
cd          descendent group ls   register rmdir     user      

Undocumented commands:
======================
help
\end{lstlisting}
In this particular case, the ``help'' command was used. It displays help on the
available functionality. Typing ``help cd'' for e.g. gives the corresponding
help for the ``cd'' command.

The most useful commands for a regular user and/or production manager are the
commands:
\begin{itemize}
  \item meta: has itself a set of sub commands, like ``show'', ``set'' or
  ``get''. Type ``help meta'' for additional information. The last argument of
  thiose commands are either a directory or a file. This command is detailed in
  sec.~\ref{sec:meta}.
  \item find: find a data set corresponding to a set of meta data, see
  sec.~\ref{sec:find}.
  \item ancestor and descendent: as the names suggests,
\end{itemize}

Some commands should {\color{red} NEVER} be used when not knowing precisely
what they do: ``register'', ``unregister'', ``rm'', ``add'', ``rmdir''. Indeed, 
those commands are misleading: they act only on the File Catalog and not on the
grid files. This can lead to having orphans on the Storage Elements, very hard if not
impossible, to recover.

\subsubsection{The meta command}\label{sec:meta}
The meta commands allow to interact with the meta data information. 

First a word of warning: as of \today, the meta data cannot be changed when set,
either from the CLI or the API, except from a direct intervention in the data
base. That means that one has to be careful when setting meta data (using ``meta
set'').

The same applies for searchable meta data fields and non searchable meta data:
they cannot be either changed or deleted (``meta index'' command). 

Searchable meta data is defined as follows:
\begin{itemize}
  \item meta index SomeMetaName metatype 
  \item meta set some/path/ SomeMetaName SomeMetaValue (for a directory)
  \item meta set some/path/file SomeMetaName SomeMetaValue (for a file)
\end{itemize}

Non searchable meta data is simply set:
\begin{itemize}
  \item meta set some/path SomeMetaName SomeMetaValue (for a directory)
  \item meta set some/path/file SomeMetaName SomeMetaValue (for a file)
\end{itemize}

~\\

Obtaining the meta data for a file or directory is done by running ``meta get
some/path'' or ``meta get some/path/file'' and returns something like:
\begin{lstlisting}
FC:/> meta get /ilc/prod/clic/1.4tev/ch1ch1/gen/00001186/
            *EvtType : ch1ch1
     *NumberOfEvents : 1000
           *Datatype : gen
             *Energy : 1.4tev
          SWPackages : whizard.SUSY_V22
            *Machine : clic
             !ProdID : 1186
\end{lstlisting}
Three types of meta data are available in tha example directory: 
\begin{itemize}
  \item inherited metadata: indicated with *. The meta data is inherited from a
  parent directory, therefore conflicting meta data setting will be rejected
  with an error.
  \item Local meta data: indicated by !. This meta data is given to sub
  directories
  \item Non searchable meta data: no specific sign (here SWPackages). This is
  always inherited. 
\end{itemize}

~\\

Finding compatible meta data is necessary: having an event type and an energy,
one might need to obtain e.g. the corresponding production IDs. The
corresponding call is ``meta tags MetaKey where key1=A key2=B'', like in the example below:
\begin{lstlisting}
FC:/> meta tags ProdID where EvtType=ch1ch1 Energy=1.4tev
Possible values for ProdID:
803
804
834
\end{lstlisting}

~\\

Showing all available meta data tags is done by ``meta show''.

~\\

Finally, the DIRAC File Catalog comes with the ``metaset'' functionality: defining a
name for a set of search fields, for quick lookup. See ``help meta'' for more
information, as there are no such example in our FC.

\subsubsection{The find command}\label{sec:find}
Here we describe briefly the ``find'' command and output. 

Once one has an idea of the fields that are relevant, finding the corresponding
set of files is done using e.g.
\begin{lstlisting}
FC:/> find ProdID=1186 EvtType=ch1ch1 Datatype=gen
Query: {'EvtType': 'ch1ch1', 'Datatype': 'gen', 'ProdID': 1186}
/ilc/prod/clic/1.4tev/ch1ch1/gen/00001186/000/ch1ch1_gen_1186_1.stdhep
/ilc/prod/clic/1.4tev/ch1ch1/gen/00001186/000/ch1ch1_gen_1186_10.stdhep
/ilc/prod/clic/1.4tev/ch1ch1/gen/00001186/000/ch1ch1_gen_1186_100.stdhep
/ilc/prod/clic/1.4tev/ch1ch1/gen/00001186/000/ch1ch1_gen_1186_101.stdhep
/ilc/prod/clic/1.4tev/ch1ch1/gen/00001186/000/ch1ch1_gen_1186_102.stdhep
/ilc/prod/clic/1.4tev/ch1ch1/gen/00001186/000/ch1ch1_gen_1186_103.stdhep
/ilc/prod/clic/1.4tev/ch1ch1/gen/00001186/000/ch1ch1_gen_1186_104.stdhep
/ilc/prod/clic/1.4tev/ch1ch1/gen/00001186/000/ch1ch1_gen_1186_105.stdhep
/ilc/prod/clic/1.4tev/ch1ch1/gen/00001186/000/ch1ch1_gen_1186_106.stdhep
\end{lstlisting}

The ``=`` condition can be replaced by logical comparison, ``>'', ``<'', ``<=``,
``>=``, ``!=''

Of course, in our FC model, there is redundancy: the ProdID itself is enough
usually, as it is the last step in the path structure. But one could want to
obtain ALL files of a given type, despite its production ID. 

\subsection{The API}
The API (Application Programing Interface) allows to use the File Catalog method
directly in one's python code. This is used in particular for file registration
at the end of a production job.

\subsection{The Web client}
The web interface is being developed and should become available soon. It will
allow for file browsing and lookup, and even direct download. It will rely on
the API described above.

\section{Conclusions}
We have described how the File Catalog was used for the CLIC CDR mass
production.
\end{document}
