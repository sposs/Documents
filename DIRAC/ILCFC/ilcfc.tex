\documentclass[11pt,a4paper]{scrartcl}
\setlength{\topmargin}{1cm}

\usepackage{lcd}
\usepackage{lcd_title,ifthen}

\usepackage{tikz}
\usetikzlibrary{arrows,shapes}

\usepackage{courier}

\usepackage[T1]{fontenc}
\usepackage{graphicx}
\usepackage{subfig}
\usepackage{lcd}
\usepackage{lcd_title,ifthen}
\usepackage{mathptmx}
\usepackage{helvet}
\usepackage{amsmath}
\usepackage{color}
\usepackage{units}
\usepackage{url}
\usepackage[pdftex,bookmarks=true,breaklinks=true,bookmarksnumbered=true,colorlinks=true,
linkcolor=webgreen,citecolor=webred,urlcolor=webblue]{hyperref}
\usepackage{breakurl}	% break hyperref urls with ps2pdf
%% link colors
\definecolor{webred}{rgb}{0.5, 0, 0}
\definecolor{webgreen}{rgb}{0, 0.5, 0}
\definecolor{webblue}{rgb}{0, 0, 0.5}
%
\usepackage[british]{babel}
\makeatletter
\def\url@myurlfontstyle{%
  \@ifundefined{selectfont}{\def\UrlFont{\sf}}{\def\UrlFont{\small\ttfamily}}}
\makeatother
\urlstyle{myurlfont}
\long\def\symbolfootnote[#1]#2{\begingroup%
\def\thefootnote{\fnsymbol{footnote}}\footnote[#1]{#2}\endgroup} 

\newlength{\capindent}
\setlength{\capindent}{0.5cm}
\newlength{\capwidth}
\setlength{\capwidth}{\textwidth}
\addtolength{\capwidth}{-2\capindent}
\newlength{\figwidth}
\setlength{\figwidth}{\textwidth}
\addtolength{\figwidth}{-2.0cm}
\newcommand{\icaption}[2][!*!,!]{\hspace*{\capindent}%
  \begin{minipage}{\capwidth}
    \ifthenelse{\equal{#1}{!*!,!}}%
      {\caption{#2}}%
      {\caption[#1]{#2}}
      \vspace*{3mm}
  \end{minipage}}

\lcdopennote{2012-009}

\usepackage{listings}
\lstloadlanguages{python}
\lstset{language=python,basicstyle=\footnotesize}
% definitions
\newcommand{\mr}{\mathrm}
\newcommand{\abs}{\vspace{12pt}}
\renewcommand{\topfraction}{0.9}
\renewcommand{\bottomfraction}{0.9}
\renewcommand{\textfraction}{0.1}
\renewcommand{\floatpagefraction}{0.85}


\begin{document}
\begin{titlepage}
%
% Print version number of document
\mydocversion
\title{DIRAC FileCatalog Structure and Content\\ in preparation of the CLIC CDR}
\author{S.~Poss}
\tikzstyle{decision} = [diamond, draw, text badly centered, node distance=2.8cm]
\tikzstyle{block} = [rectangle, draw, text centered, rounded corners, node distance=2.2cm]
\tikzstyle{autoblock} = [rectangle, draw, text centered, rounded corners, node distance=2.2cm]
\tikzstyle{line} = [draw, -triangle 90]
\tikzstyle{dline} = [draw, dashed, -triangle 90]

%\maketitle
\abstract{This document presents how the File Catalog was organized for the
CLIC CDR. It also gives the searchable meta data information currently
available. The last sections are dedicated to the interaction with the File
Catalog.}

\tableofcontents
\end{titlepage}

\section{Introduction}
The CLIC CDR mass production used the DIRAC File Catalog (FC)\cite{diracfc} as
replica and meta data catalog. This proved to be an efficient solution, and made bookkeeping
straight forward for the data manager. 

This File Catalog allows for storing the files in different directories,
exactly like any file system. The Logical File Names (LFN) allows to obtain the 
Physical File Names (PFN) by adding the server details (srm end point and base 
path) of one Storage Element (SE) containing the file. This was chosen in order 
to facilitate FileCatalog - SE consistency checks.

As for the meta data, there are two kinds: searchable and non-searchable.
The difference is of pure practical purposes, as in principle, any meta data
can be made searchable. Chosing wisely with meta data is searchable is the key
to scalability. Meta data can be attached to directories and/or to files, both
for searchable and non-searchable. Obviously, searchable meta data at the file 
level is much less efficient than at the directory level, therefore only at the
directory level some were used for the CLIC CDR. Non searchable meta data at
the file level was added at a later stage, and holds some information related 
to the file.

The File Catalog also holds ancestor-daughter relationships between the files,
allowing to navigate easily from one step of the production to the
next, for instance simulation $\to$ reconstruction.

In the following sections, we first show the chosen path and file name
conventions used, and then we discuss the available searchable meta data. The
last sections are intended to be viewed as a manual.

\section{Organisation of the CLIC CDR data}
This section describes how the path structure of the FC was designed. 

The base element is by convention the VO, in this case
``\lstinline[language=bash]|/ilc|''. This allows to have multiple VOs using the
same File Catalog. For example, we also support CALICE users with the same DIRAC
instance. The corresponding data is available under
``\lstinline[language=bash]|/calice|''.

The second level distinguishes ``production'' and ``user'' activities, so their
paths are the following:
\begin{itemize}
  \item \lstinline[language=bash]|/ilc/prod| for production,
  \item \lstinline[language=bash]|/ilc/user| for user data.
\end{itemize} 
In the following, only the production directories are described. 

As sinmulation for two accelerator concepts are being performed under the same
VO, the next level is the machine, ``\lstinline[language=bash]|ilc|'' or
``\lstinline[language=bash]|clic|''. In the case of the
``\lstinline[language=bash]|ilc|'' directory,  the data structure available
follows the ILC DBD conventions for the generation.

For the clic structure, the sub directory are organized like this:\\
\lstinline[language=bash]|Energy/EvtType/gen/ProdID/000| for generator level,
or\\
\lstinline[language=bash]|Energy/EvtType/DetectorType/Datatype/ProdID/000|
for simulated and reconstructed data. The individual elements are defined as follows:
\begin{itemize}
  \item Energy: energy of the machine: ``\lstinline|3tev|'',
  ``\lstinline|1.4tev|'', ``\lstinline|500gev|'', or ``\lstinline|350gev|'';
  \item EvtType: the event type, like ``\lstinline|WW|'';
  \item DetectorType: as the mane suggests, either ``\lstinline|ILD|'' or
  ``\lstinline|SID|'';
  \item Datatype: either ``\lstinline|SIM|'', ``\lstinline|REC|'', or
  ``\lstinline|DST|'';
  \item ProdID: the internal production ID of DIRAC, as we are using its
  Transformation System. It has leading 0 such that the length of the string is
  8 chars long.
  \item The last \lstinline|000| is the taskID\%1000: most Storage Elements and
  the File Catalog suffer from performance loss when having too many files in the same
  directory. So those sub directories contain at most 1000 files.
\end{itemize}

So the full path structure is then the following e.g.:\\
\lstinline[language=bash]|/ilc/prod/clic/1.4tev/ch1ch1/gen/00001186/000/| for
generator level data, or \\
\lstinline[language=bash]|/ilc/prod/clic/1.4tev/ch1ch1/SID/REC/00001189/000/|
for REC files produced with the SID detector.

~\\

The file names also follow conventions: They are built using the minimal
set of information needed to obtain the most important information: event type,
data type, production ID, task ID in that production, and, in case of split
files, a sub file index. For example, in the files produced in the generator
step for the sample above, ch1ch1, the file names are:
\lstinline[language=bash]|ch1ch1_gen_1186_96.stdhep|.  From this file name,
one can go to the Production Monitoring page, find the production 1186, check
it's properties, look at  the corresponding jobs, and find the task 96 (not the 
same as the job ID) by looking at  the job names\footnote{Here the job name is 
00001186\_00000096.}, and obtain the job's details. In some future, log files 
for a particular job should also be made available from the web portal, as the 
functionality is available.

In case of splitting at the lcio level, the file name is e.g.
\lstinline[language=bash]|ch1ch1_sim_1188_98-3-200.slcio|. The additional
elements denote the file index in the splitting and the number of events in the
file, respectively. This is a convention coming from the ``lcio split'' utility. 

In case of stdhep split, the convention is different, as it only adds a split
file index, as in \lstinline[language=bash]|ch1ch1_gen_1186_96_1.stdhep|. We
have no files in the FC produced with the stdhep splitter. 

\section{Meta data used for the CLIC CDR}
The meta data is separated in two categories: searchable and non searchable.
They are presentesd in the following sections.

\subsection{Searchable meta data}
The aim of searchable meta data is to obtain a data set efficiently using
key=value type identifiers\footnote{See section~\ref{sec:find} for possible
operators.}. This data set is to be used either as input to productions or for a
user to run analysis. In both case, a hierarchisation  of the information is needed. In practice, it was chosen to
use mostly the same information as the path structure. Indeed, the searchable
meta data used is attached to directories. Below is the list of meta data keys 
that can be used for setting and reading:
\begin{itemize}
  \item EvtType (string): event type, same as in path
  \item Datatype (string): data type: \lstinline|gen|, \lstinline|SIM|,
  \lstinline|REC|, \lstinline|DST|, same as in path
  \item Energy (string): same as in path
  \item DetectorType (string): same as in path
  \item Machine (string): same as in path (\lstinline|clic| or \lstinline|ilc|)
  \item ProdID (int): same as in path, but the leading 0 removed
  \item DetectorModel (string): not used yet, but can hold for instance
  \lstinline|sidloi3|
  \item MachineParams (string): not used, but meant to hold ilc machine
  parameters name (\lstinline|b1_ws| for example)
  \item NumberOfEvents (int): common number of events in each file in the sub
  directories. This is used to define some type of productions (Mokka), but is
  not used in said production, as this number is now stored at the file level
  (see below).
  \item Luminosity (int): same as NumberOfEvents, used to obtain the
  corresponding cross section before is was stored for every file.
  \item Owner (string): only used for users' files, under \lstinline|/ilc/user|
  \item JobType (string): not used, was meant to store the job type
  (generation, simulation, reconstruction, etc.) but overlaps with the other
  metadata
  \item runnumber (int): not used and not needed (legacy from initial
  development)
  \item BXoverlayed (float): not used, was meant to be filled when having
  overlay, but the overlay info can be obtained from the Transformation system. 
  \item StartDate (date): not used, but is meant to hold the start date of the
  production. We do not need it.
\end{itemize}
Using this meta data is described in section~\ref{sec:usingfc}. 

It can be noted that this list can be extended as much as needed, but the
position of the meta data in the name space must be chosen carefully in order
not to conflict with the existing meta data.

\subsection{Non searchable meta data}
This feature was added at a late stage compared to the CLIC CDR mass production
time scale. Therefore, most data does not use this type of meta data. But the
data produced for the CDR volume 3 makes use of it. 

Any string meta data can be used, there is only a restriction in the length of
the string, not more than 255 characters are allowed. File annotations are among
the future improvements and they will not have size limitations.

As this is non searchable, any key can be attached to any directory/file. As a
convention, we use the following at directory level:
\begin{itemize}
  \item SWPackage: software packages used for a given production, for example\\
  \lstinline|whizard.SUSY_V22|, or
  \lstinline|lcsim.CLIC_CDR;SLICPandora.V6;lcsim.CLIC_CDR|.
\end{itemize}

~\\

At the file level, more meta data is available:
\begin{itemize}
  \item Luminosity: for whizard files only
  \item NumberOfEvents: for any file
  \item AdditionalInfo: any file in principle, but we set some only
  for files that are produced by Whizard, and is by convention a
  DEncode object: \\ 
  \begin{lstlisting}
'ds10: xsection ds4:sum ds14: err_xsection f0.201es10:
xsection f213.67169es9: fraction f100.0eeee'
  \end{lstlisting}
  This utility is found under \lstinline|DIRAC.Core.Utilities.DEncode|.
\end{itemize}
Anything can be added, but because of the string length limitation, only 255
chars are allowed (important for the last field).

~\\

Additionally, the ancestor-daughter relationship is stored allowing to obtain
the full history of a given file. Using this feature will be described in the next
section.

~\\

In this section we presented the meta data used during the CLIC CDR mass
production. They can be extended to cover new needs. The number of possible tags
for searchable meta data should be limited for human usefulness reasons. The non
searchable meta data will be completed with file annotations in the future, 
allowing more details to be added.

The next section presents how the meta data information discussed here can be
accessed and modified.

\section{Interacting with the File Catalog}\label{sec:usingfc}
There are several ways to interact with the file catalog: the CLI, the API, and
in a near future, a web client.

\subsection{The command line interface}
The command line interface (CLI) can be invoked by typing in a terminal
\emph{dirac-dms-filecatalog-cli}. This gives a shell like interface, where
commands are to be entered, for example:
\begin{lstlisting}
FC:/> help

Documented commands (type help <topic>):
========================================
add         chgrp      exit  guid meta     replicas  rmreplica 
ancestor    chmod      find  id   mkdir    replicate size      
ancestorset chown      get   lcd  pwd      rm        unregister
cd          descendent group ls   register rmdir     user      

Undocumented commands:
======================
help
\end{lstlisting}
In this particular case, the ``\lstinline|help|'' command was used. It displays
help on the available functionality. Typing ``\lstinline|help cd|'' for e.g. gives the
corresponding help for the ``\lstinline|cd|'' command.

The most useful commands for a regular user and/or production manager are the
commands:
\begin{itemize}
  \item meta: has itself a set of sub commands, like ``\lstinline|show|'',
  ``\lstinline|set|'' or ``\lstinline|get|''. Type ``\lstinline|help meta|'' for
  additional information. The last argument of those commands are either a 
  directory or a file. This command is detailed in sec.~\ref{sec:meta}.
  \item find: find a data set corresponding to a set of meta data, see
  sec.~\ref{sec:find}.
  \item ancestor and descendent: as the names suggests.
\end{itemize}

Some commands should remain exclusive to expert users: ``\lstinline|register|'',
``\lstinline|unregister|'', ``\lstinline|rm|'', ``\lstinline|add|'',
``\lstinline|rmdir|''. Indeed, those commands are misleading: they act only on the
File Catalog and not on the grid files. This can lead to having orphans on the Storage Elements, very hard if not impossible,
to recover.

\subsubsection{The meta command}\label{sec:meta}
The meta commands allow to interact with the meta data information. 

First a word of warning: as of \today, the meta data cannot be changed when set,
either from the CLI or the API, except from a direct intervention in the data
base. That means that one has to be careful when setting meta data (using
``\lstinline|meta set|'').

The same applies for searchable meta data fields and non searchable meta data:
they cannot be either changed or deleted (``\lstinline|meta index|'' command). 

Searchable meta data is defined as follows:
\begin{itemize}
  \item \lstinline|meta index SomeMetaName metatype| defines a new field
  \item \lstinline|meta set some/path/ SomeMetaName SomeMetaValue| for a
  directory,
  \item \lstinline|meta set some/path/f SomeMetaName SomeMetaValue| for a
  file
\end{itemize}

Non searchable meta data is simply set:
\begin{itemize}
  \item \lstinline|meta set some/path SomeMetaName SomeMetaValue| for a
  directory,
  \item \lstinline|meta set some/path/f SomeMetaName SomeMetaValue| for a file
\end{itemize}

~\\

Obtaining the meta data for a file or directory is done by running
``\lstinline|meta get some/path|'' or ``\lstinline|meta get some/path/file|''
and returns something like:
\begin{lstlisting}
FC:/> meta get /ilc/prod/clic/1.4tev/ch1ch1/gen/00001186/
            *EvtType : ch1ch1
     *NumberOfEvents : 1000
           *Datatype : gen
             *Energy : 1.4tev
          SWPackages : whizard.SUSY_V22
            *Machine : clic
             !ProdID : 1186
\end{lstlisting}
Three types of meta data are available in tha example directory: 
\begin{itemize}
  \item inherited meta data: indicated with *. The meta data is inherited from a
  parent directory, therefore conflicting meta data setting will be rejected
  with an error.
  \item Local meta data: indicated by !. This meta data is given to sub
  directories
  \item Non searchable meta data: no specific sign (here SWPackages). This is
  always inherited. 
\end{itemize}

~\\

Finding compatible meta data is necessary: having an event type and an energy,
one might need to obtain e.g. the corresponding production IDs. The
corresponding call is \\
``\lstinline|meta tags MetaKey where key1=A key2=B|'',
like in the example below:
\begin{lstlisting}
FC:/> meta tags ProdID where EvtType=ch1ch1 Energy=1.4tev
Possible values for ProdID:
803
804
834
\end{lstlisting}

~\\

Showing all available meta data tags is done by ``\lstinline|meta show|''.

~\\

Finally, the DIRAC File Catalog comes with the ``\lstinline|metaset|''
functionality: defining a name for a meta data query, for quick look up.
See ``\lstinline|help meta|'' for more information, as there are no such example
in our FC.

\subsubsection{The find command}\label{sec:find}
Here we describe briefly the ``find'' command and its output. 

Finding the set of files corresponding to a certain set of meta data is done
for example using:
\begin{lstlisting}
FC:/> find ProdID=1186 EvtType=ch1ch1 Datatype=gen
Query: {'EvtType': 'ch1ch1', 'Datatype': 'gen', 'ProdID': 1186}
/ilc/prod/clic/1.4tev/ch1ch1/gen/00001186/000/ch1ch1_gen_1186_1.stdhep
/ilc/prod/clic/1.4tev/ch1ch1/gen/00001186/000/ch1ch1_gen_1186_10.stdhep
/ilc/prod/clic/1.4tev/ch1ch1/gen/00001186/000/ch1ch1_gen_1186_100.stdhep
/ilc/prod/clic/1.4tev/ch1ch1/gen/00001186/000/ch1ch1_gen_1186_101.stdhep
/ilc/prod/clic/1.4tev/ch1ch1/gen/00001186/000/ch1ch1_gen_1186_102.stdhep
/ilc/prod/clic/1.4tev/ch1ch1/gen/00001186/000/ch1ch1_gen_1186_103.stdhep
/ilc/prod/clic/1.4tev/ch1ch1/gen/00001186/000/ch1ch1_gen_1186_104.stdhep
/ilc/prod/clic/1.4tev/ch1ch1/gen/00001186/000/ch1ch1_gen_1186_105.stdhep
/ilc/prod/clic/1.4tev/ch1ch1/gen/00001186/000/ch1ch1_gen_1186_106.stdhep
\end{lstlisting}

The ``=`` condition can be replaced by logical comparison, ``>'', ``<'', ``<=``,
``>=``, ``!='', or ``= a,b,c'' to select all element in the set (a,b,c). Using
``!= a,b,c'' excludes all elements in the set (a,b,c).

There is some redundancy in our FC model: the ProdID itself is usually enough,
as it is the last step in the path structure. As an example, one could want to
obtain ALL files of a given type, despite its production ID.

\subsection{The application programing interface}
The Application Programing Interface (API) allows to use the File Catalog method
directly from python code. This is used in particular for file registration
at the end of a production job.

There are several commands that are relevant for developers, and potentially
also for users. The first set is related to adding and setting meta data, and
the second set is dedicated to reading said meta data. Those sets are described
below.

\subsubsection{Adding and setting meta data}
In this section, we only discuss how to set a searchable meta data value and
adding a non searchable meta data. The reason is that to add a searchable meta
data, the CLI provides the necessary tools. Moreover, non searchable meta data
with a job specific key (JobID for example) should not be used, because it
would lead to as many fields as there are files.

The setting of meta data is done the following way:
\begin{lstlisting}
from DIRAC.Resources.Catalog.FileCatalogClient import FileCatalogClient

fc = FileCatalogClient()

meta = {}
meta['SomeKey']=somevalue

#for a directory
d = '/some/path/' 
res = fc.setMetadata(d,meta)
#for a file
f = '/some/path/file'
res = fc.setMetadata(f,meta)

if not res["OK"]:
  print res['Message']
\end{lstlisting}
In both case, the directory or the file must be already available in he Catalog. 
The meta data set for a given directory is only inherited to the descendents.
That means it is not possible to overwrite the value of a meta data in a sub
directory.

For completeness, we show in section~\ref{sec:addingfiles} how files are
added to the FC.

This setting of meta data works both for setting a searchable meta data value
and for adding a non searchable meta data.

The return value \lstinline|res| is a dictionary with at least one key
\lstinline|OK| that is boolean. If false, there is another key
\lstinline|Message| that holds the error message.

~\\

Setting ancestor-daughter relationships is done using the following:
\begin{lstlisting}
from DIRAC.Resources.Catalog.FileCatalogClient import FileCatalogClient

fc = FileCatalogClient()
f = '/some/path/file'

res = fc.addFileAncestors({f:{'Ancestors':list_of_ancestors}})
if not res["OK"]:
  print res['Message']
\end{lstlisting}

Of course, since the argument of the function \lstinline|addFileAncestors| is a
dictionary, multiple files can have their ancestors defined in one call.

\subsubsection{Reading the meta data}
There are also several API functions to interact with the FC to read
information. 

To obtain the directory meta data, which is usually searchable meta data, one
should use the following:
\begin{lstlisting}
from DIRAC.Resources.Catalog.FileCatalogClient import FileCatalogClient
fc = FileCatalogClient()

d = '/some/path/' 

res = fc.getDirectoryMetadata(d)
if res['OK']:   
  tags= res['Value']
  print tags['SomeValidKey']
\end{lstlisting}
This returns the meta data of the directory \lstinline|d| as well as the
inherited one. One way to distinguish between the different types of meta data
is to use the following:
\begin{lstlisting}
res = fc.getDirectoryMetadata(d)
if res['OK']:   
  tags= res['Value']
  metatypes = res['MetadataType']
  if 'SomeValidKey' in tags and 
     metatypes['SomeValidKey'] == 'OwnParameter':
    print "SomeValidKey is a non searchable metadata"
\end{lstlisting}

For the file level meta data, the following returns the complete set, with no
distinction between searchable and non searchable meta data.
\begin{lstlisting}
from DIRAC.Resources.Catalog.FileCatalogClient import FileCatalogClient

fc = FileCatalogClient()
f = '/some/path/file'
res = fc.getFileUserMetadata(f)
if res['OK']:   
  tags = res['Value']
  print tags['SomeValidKey']
\end{lstlisting}

~\\

Obtaining the ancestors and descendants can also be done using the API, with the
following:
\begin{lstlisting}
from DIRAC.Resources.Catalog.FileCatalogClient import FileCatalogClient

fc = FileCatalogClient()
flist = ['/some/path/file1','/some/path/file2']
res = fc.getFileAncestors(flist,1)#or 2, or 3, or 4
if not res['OK']:
  print res['Message']
  exit()
#otherwise, res has 2 keys 'Successful' and 'Failed'  
for item in flist:
  if res['Value']['Successful'].has_key(item):
    ancestors_at_level_N = res['Value']['Successful'][item].keys()
\end{lstlisting}
The argument of the function is the depth to search, 1 is the mother, 2, the
grand mother, 3, the grand grand mother, etc. One can access the ancestor chain
in one go (e.g. here up to 3):
\begin{lstlisting}
from DIRAC.Resources.Catalog.FileCatalogClient import FileCatalogClient

fc = FileCatalogClient()
flist = ['/some/path/file1','/some/path/file2']
res = fc.getFileAncestors(flist,[1,2,3])
if not res['OK']:
  print res['Message']
  exit()
#otherwise, res has 2 keys 'Successful' and 'Failed'  
for lfn,ancestorDict in res['Value']['Successful'].items():
  for ancestor,dep in ancestorDict.items():
    print "ancestor", ancestor, " at depth", dep
\end{lstlisting}

There are the equivalent functions for the descendents:
\begin{lstlisting}
res = fc.getFileDescendents(flist,1)
#or
res = fc.getFileDescendents(flist,[1,2,3])
\end{lstlisting}

Finally, obtaining files corresponding to a set of meta data is done using the
following:
\begin{lstlisting}
\begin{lstlisting}
from DIRAC.Resources.Catalog.FileCatalogClient import FileCatalogClient
fc = FileCatalogClient()
meta = {}
meta['SomeKey'] = SomeValue
meta['SomeOtherKey'] = {">=":10}
res = fc.findFilesByMetadata(meta)
if res['OK']:
  lfns = res['Value']
  print "Found %s files"%len(lfns)
\end{lstlisting}
Notice the structure used to specify arithmetic operators in the meta data
query. In that case, the union operator is \lstinline|{'in':[a,b]}|, or
\lstinline|{'nin':[a,b]}| for exclusion.

\subsection{The Web client}
The web interface is being developed and should become available soon. It will
allow for file browsing and file look up, and even direct download. It will rely
on the API described above.

\section{Adding files to the catalog}\label{sec:addingfiles}
This brief section shows how files are added to the catalog. They are not
supposed to be registered alone, instead they are supposed to be registered when
uploaded to the Storage.

The upload and registration of a file is done using the following:
\begin{lstlisting}
from DIRAC.DataManagementSystem.Client.ReplicaManager import ReplicaManager

rm = ReplicaManager()

res = rm.putAndRegister("/ilc/some/path/file","localfile","SomeStorage")
if not res['OK']:
  print res['Message']
\end{lstlisting}
The first argument of the function is the LFN (starting with \lstinline|/ilc|
for us), the second is the path to the local file, and SomeStorage is one of the valid
storage element. 

~\\

Replication to another storage element is done using the following:
\begin{lstlisting}
res = rm.replicateAndRegister("/ilc/some/path/file","SomeStorage")
\end{lstlisting}

~\\

In the future, the \lstinline|putAndRegister| command will also take the meta
data information.

\section{Conclusions}
We have described how the File Catalog was used for the CLIC CDR mass
production. The meta data fields that are searchable were chosen to allow the
minimal number of keys needed to access a given set of files. Additional non
searchable meta data were added to allow storage of how the files are obtained,
currently only the software version, as well as file specific information like
the number of events or the cross section. We have shown that adding and
reading new fields can be done using several tools requiring minimal user input.

Several improvements will be added by the DIRAC developers to facilitate and
secure the FC usage. In particular, the sensitive features of the CLI will be
made unavailable to generic users. Changing meta data should also be allowed.
The API will also have several changes, to harmonize the features, in particular
the return values. Finally, the Web client will be made available to the users.

\begin{thebibliography}{9}
\bibitem{diracfc} A. Tsaregorotsev et al., DIRAC: A Community Grid Solution,
Journal of Physics: Conference Series 119 (2008) 062048
\end{thebibliography}
\end{document}
