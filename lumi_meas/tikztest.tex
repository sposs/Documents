%\documentclass{article}
\documentclass{beamer}
%\usepackage{beamerarticle}
\usepackage{tikz}
\usetikzlibrary{arrows,shapes}

\author{S.Poss and S.Sailer}
\title{Luminosity spectrum measurement}
\subtitle{First results}

\begin{document}
\begin{frame}
\titlepage
\end{frame}
\tikzstyle{decision} = [diamond, draw, text badly centered, node distance=2.8cm]
\tikzstyle{block} = [rectangle, draw, text centered, rounded corners, text width=1.9cm, node distance=2.2cm]
\tikzstyle{autoblock} = [rectangle, draw, text centered, rounded corners, node distance=2.2cm]
\tikzstyle{line} = [draw, -triangle 90]
\tikzstyle{dline} = [draw, dashed, -triangle 90]
%\tikzstyle{cloud} = [draw, ellipse,fill=red!20, node distance=3cm,minimum height=2em]
\section{Lumi spectrum}
\begin{frame}
\frametitle{Luminosity spectrum}
Why it's important to know it:
\begin{itemize}
\item cross section measurements: Higgs, etc.
\item mass measurements: slepton analysis, etc.
\end{itemize}
What we want to ``measure'': a parametrisation that describes the luminosity
spectrum.
\begin{itemize}
\item need a correct model, based on assumptions and existing Monte Carlo
samples
\item need a framework/procedure for parameter estimation
\item need data 
\end{itemize}
\end{frame}
\section{Obtaining data}
\begin{frame}
\frametitle{Obtaining data}
\begin{figure}[h]
\begin{tikzpicture}[scale=0.8,auto]
\matrix [column sep=7mm, row sep=4mm,ampersand replacement=\&]
{
%row1
\uncover<1->{\node [block] (money) {Get money};} \&
\uncover<2->{\node [block] (build) {Build detector};} \&
\uncover<3->{\node [block] (recdata) {Take Bhabha data};} \&
\uncover<4->{\node [autoblock, color=red] (rec) {Reconstruction};} \\
%row2
~ \&
\uncover<5->{\node [block] (gen) {Generation using GP and BHWide};} \&
\uncover<6->{\node [autoblock] (sim) {Simulate};} \&
~\\
};
\uncover<2->{\path [line] (money) -- (build);}
\uncover<3->{\path [line] (build) -- (recdata);}
\uncover<4->{\path [line] (recdata) -- (rec);}
\uncover<6->{\path [line] (gen) -- (sim);}
\uncover<7->{\path [line] (sim) -| (rec);}
\end{tikzpicture}
\end{figure}
\end{frame}
\section{Chosen model}
\begin{frame}
\frametitle{Chosen model}
Several components to account for:
\begin{itemize}
  \item Beam delivery system (BDS)
  \item Beam strahlung
\end{itemize}
\end{frame}

\begin{frame}
\frametitle{Chosen model: BDS}
Use BetaFunction as a model:\\
\begin{tikzpicture}[auto]
\node[draw, node distance=2.2cm]{\includegraphics[width=10cm]{DeltaE_electron_fit_BetaFunction}};
\end{tikzpicture}
\end{frame}

\begin{frame}
\frametitle{Chosen model: BDS}
Use BetaFunction as a model:\\
\begin{tikzpicture}[auto]
\node[draw, node distance=2.2cm]{\includegraphics[width=10cm]{DeltaE_electron_fit_BetaFunction_regions}};
\node[block]{Tails modelised by gaussian with fixed width.};
\end{tikzpicture}

\end{frame}

\begin{frame}
\frametitle{Chosen model: Beam strahlung}
Use the same working hypothesis as Andr\'e's thesis: Beta Function. No plot to
show as data is missing.
\end{frame}

\section{Methods}
\begin{frame}
\frametitle{Methods}
2 methods are available:
\begin{itemize}
\item Simple fit: dumb and easy, obvious method, but not CPU wise,
\item Reweighting fit: smart and elegant, not so obvious, but clearly affordable CPU wise.
\end{itemize}
~\\
Definition: \alert{$[p]_N$ is a set of parameter values} at the $N^{th}$
iteration, not a set of parameterization: all iterations use the SAME parametrization.
\end{frame}
\section{First method: Simple fit}
\begin{frame}
\frametitle{First method: Simple fit}
\begin{figure}[h]
  \begin{tikzpicture}[scale=2.8,auto]
\uncover<1-2>{\node [autoblock] (start) {\scriptsize Start: $[p]_0$};}
\uncover<2->{\node [block, below of=start, color=blue] (gen) {\scriptsize Generate Event with $[p]_N$: E1,E2};}
\uncover<3-10,12->{\node [autoblock, right of=gen] (bhwide) {\scriptsize BHWide};}
\uncover<4-10,13->{\node [autoblock, right of=bhwide] (sim) {\scriptsize Simulation};}
\uncover<5-10,14->{\node [autoblock, right of=sim] (rec) {\scriptsize Reconstruction};}
\uncover<6-10,15->{\node [block, below of=rec, color=red] (compare) {\scriptsize Compare with data: $\chi^2$};}
\uncover<7-10,16->{\node [decision, left of=compare, color=violet] (match)
{\scriptsize Minimum?};} \uncover<8-10,17->{\node [autoblock, below of=match] (done) {\scriptsize Done!};}
\uncover<9-11,18->{\node [block, below of=gen, color=magenta] (minim) {\scriptsize Minimizer: $[p]_N \rightarrow [p]_{N+1}$};}
\uncover<1-2>{\path [line] (start) -- (gen);}
\uncover<3-10,12->{\path [line] (gen) -- (bhwide);}
\uncover<4-10,13->{\path [line] (bhwide) -- (sim);}
\uncover<5-10,14->{\path [line] (sim) -- (rec);}
\uncover<6-10,15->{\path [line] (rec) -- (compare);}
\uncover<7-10,16->{\path [line] (compare) -- (match);}
\uncover<8-10,17->{\path [line] (match) -- node [near start] {\scriptsize yes} (done);}
\uncover<9-10,18->{\path [line] (match) -- node [near start] {\scriptsize no} (minim);}
\uncover<10-11,19->{\path [line] (minim) -- (gen);}
%\node (start) at (0,2) [draw] {Start};
%\node (gen) at (0,1) [draw] {Generate Event with $[P]_N$: E1,E2};
\end{tikzpicture}
\end{figure}
\end{frame}
\begin{frame}
\frametitle{First method: Simple fit}
Pros:
\begin{itemize}
\item logical
\item straight-foward to set up
\end{itemize}
Cons:
\begin{itemize}
\item Takes forever
\item all MC samples need to be reconstructed (1 reco per iteration)
\end{itemize}
~\\
\alert{Need better idea.}
\end{frame}

\section{Second method: Reweighting fit}
\begin{frame}
\frametitle{Understanding the reweighting fit}
\begin{itemize}
  \item a \alert{distribution} of an observable $=$ \alert{``probability''} for
  an event to happen with a given observable value (observable can be e.g. $\{EB1,EB2\}$ for
  a given event)%, in fact 1 event
  %$=\{E1,E2,\theta_1,\theta_2,E_{ECAL1},E_{ECAL2}\}$)
  \item if said distribution is built from a set of parameters' values
  $[p]$, then probabilities can be computed from that set
  \item Changing $[p]\to [p]'\Rightarrow$ changes of the probabilities. This
  change can be seen in the ratio $\frac{P(\{EB1,EB2\},
  [p]')}{P(\{EB1,EB2\},[p])}$. This is the weight applied to every event.
  \item One event is not only represented by $\{EB1,EB2\}$ but by
  $\{EB1,EB2,\theta_1,\theta_2,E_{ECAL1},E_{ECAL2},etc.\}$ where the
  $\{\theta_1,\theta_2,E_{ECAL1},E_{ECAL2},etc.\}$ set are reconstructed
  (measured) observables
\end{itemize}
\end{frame}
\begin{frame}
\frametitle{Second method: reweighting fit}
%Shown in fig.~\ref{fig:second}
%begin{figure}[h]
\begin{tikzpicture}[scale=.8,auto, remember picture]
\matrix [column sep=5mm,row sep=4mm,ampersand replacement=\&]{
%row1
\uncover<1>{\node [autoblock] (start) {\scriptsize Start};} \& 
~ \& 
~ \&
~ \\
%row2
\uncover<2->{\node [block, color=blue] (gen) {\scriptsize Generate Event with $[p]_0$: $E1$, $E2$, $P(E1,E2;[p]_0)$};} \&
\uncover<3-5>{\node [autoblock, node distance=2.7cm] (bhwide) {\scriptsize BHWide};} \&
\uncover<4-5>{\node [autoblock, node distance=2.5cm] (sim) {\scriptsize Simulation};} \&
\uncover<5->{\node [autoblock, node distance=2.7cm] (rec) {\scriptsize Reconstruction};} \\
%row3
\uncover<6->{\node [block, color=magenta] (minim) {\scriptsize Minimizer: $[p]_N \rightarrow [p]_{N+1}$};} \&
\uncover<7-13,15->{\node [block, text width=2.2cm, node distance=2.7cm] (compute) {\scriptsize Compute $P(E1,E2;[p]_{N+1})$ for all events};} \&
\uncover<8-13,16->{\node [autoblock, node distance=2.8cm] (weight) {\scriptsize $w = \frac{P(E1,E2;[p]_{N+1})}{P(E1,E2;[p]_{0})}$};} \&
\uncover<9-13,17->{\node [block, node distance=2.7cm] (weightrec) {\scriptsize Weight every event with its weight $w$};} \\
%row4
~ \& 
~ \& 
\uncover<11-13,19->{\node [decision, color=violet] (match) {\scriptsize
Minimum?};} \& \uncover<10-13,18->{\node [block, color=red] (compare) {\scriptsize Compare with data: $\chi^2$};}\\
%row5
~ \& 
~ \& 
\uncover<12-13,20->{\node [autoblock] (done) {\scriptsize Done!};} \& 
~ \\
};
\begin{scope}
\uncover<1>{\path [line] (start) -- (gen);}
\uncover<3-5>{\path [line] (gen) -- (bhwide);}
\uncover<4-5>{\path [line] (bhwide) -- (sim);}
\uncover<5>{\path [line] (sim) -- (rec);}
\uncover<6>{\path [line] (gen) -- (minim);}
\uncover<7-13,15->{\path [line] (minim) -- (compute);}
\uncover<8-13,16->{\path [line] (compute) -- (weight);}
\uncover<9-13,17->{\path [line] (weight) -- (weightrec);}
\uncover<10-13,18->{\path [line] (weightrec) -- (compare);}
\uncover<11-13,19->{\path [line] (compare) -- (match);}
\uncover<12-13,20->{\path [line] (match) -- node [near start] {\scriptsize yes} (done);}
\uncover<13,21->{\path [line] (match) -| node [very near start] {\scriptsize no} (minim);}
\uncover<9-13,17->{\path [dline] (rec) -- node [midway] {\scriptsize use} (weightrec); }
\end{scope}
\end{tikzpicture}
%\end{figure}
\end{frame}
\begin{frame}
\frametitle{Second method: reweighting fit}
As for the moment, data and MC samples share the same generator, simulation and
reconstruction, there are no fondamental difference in the behavior: e.g. same
detector resolution $\Rightarrow$ non need to account for those effects for the
moment, and one can validate the method by looking only at generator level data.
\end{frame}

\section{Generated samples}
\begin{frame}
\frametitle{Generated samples: BDS distribution}
\includegraphics[width=10cm]{MCBeamSpread}
\end{frame}

\section{So far}
\begin{frame}
\frametitle{So far}
We have:
\begin{itemize}
\item framework
\item MC production tools: generator level
\item Guinea pig level data: generator level
\end{itemize}
Simulation and reconstruction will be done later, when some results are obtained. 

~\\

What we don't have yet:
\begin{itemize}
\item a model that works on generator level data.
\end{itemize}
\end{frame}
\end{document}
