\documentclass{article}

\begin{document}
\section{Luminosity measurement at CLIC}
In order to assess the systematic error on the measurements described later in
this document, it is mandatory to provide an estimate of the luminosity
spectrum. For this purpose, a procedure based on a so-called reweighting fit \cite{andre}
was developed. The fit is performed against a sample produced using Guinea
Pig\cite{guineaspig}, that will be refered to as ``data'' in the rest of this
section. The advantage of such fitting method is that it does not require to
simulate and reconstruct many times a given sample, as it only changes the individual event weights.

The measurements performed has three steps. 

Firsly, one generates a set of pairs of beam energies assuming a model.
  This model is defined by analysing the different contributions that enter the
  luminosity spectrum, namely the beam energy spread and the beamstrahlung. The
  former is shown in figure while the later is modeled using the same hypothesis
  as in \cite{andre}. To account for the different components visible in fig.
  they are individually mathematically described using several beta functions.
  The final combinaison is given in \cite{lumimeas_toappear} and has 20 free
  parameters. The fit is then performed against the Guinea Pig sample. 
  
Secondly, the previously generated sample is used as input to BHWide,
  the large angle Bhabha scattering generator \cite{bhwide}, to produce bhabha
  events. The relevant quantities considered are the energies of the electron
  and positron, and the acollinearity. Those quantitites are represented in fig. 
  
Finaly, due to time constrains for the publication of this report, it
  was chosen not to perform the full simulation/reconstruction chain, but
  instead smear the energy of the reconstructible leptons, according to the
  formula REF. The distributions of the energy with and without smearing are
  given in fig.


 Use reweighting method (Andre et al.) 

MC sample created assuming some model

1) Fit of the MC sample vs GP

2) Looking at reconstructible final states: BHWide as generator for Bhabha
scattering at large angles

3) energy smearing, assumed 15\%/sqrt(E) 

Model description: Beta function for the tails (plot?) and beta function + gauss
smearing for the peak (plot)

2D plots for E1vsE2 at MC level/ GP

3plots e1,e2, effsqrt for BHWide sample

4 final plots spectra/peakonly/diffvsGP/peakonly



\begin{thebibliography}{99}
\bibitem{andre}{Andre Sailer's diploma thesis}
\end{thebibliography}
\end{document}