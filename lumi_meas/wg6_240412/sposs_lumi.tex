%\documentclass{article}
\documentclass[handout]{beamer}
%\usepackage{beamerarticle}
\usepackage{tikz}
\usetikzlibrary{arrows,shapes}

\author{S.Poss and A.Sailer}
\title{Luminosity spectrum measurement}
\subtitle{First results}

\mode<presentation>
{
   \setbeamertemplate{navigation symbols}{}
   \setbeamertemplate{footline}[frame number] 
}

\AtBeginSection[]
{
\begin{frame}<beamer>
\frametitle{Outline}
\tableofcontents[currentsection,currentsubsection]
\end{frame}
}


\begin{document}
\begin{frame}
\titlepage
\end{frame}
\begin{frame}
\frametitle{Outline}
\tableofcontents
% You might wish to add the option [pausesections]
\end{frame}
\tikzstyle{decision} = [diamond, draw, text badly centered, node distance=2.8cm]
\tikzstyle{block} = [rectangle, draw, text centered, rounded corners, text width=1.9cm, node distance=2.2cm]
\tikzstyle{autoblock} = [rectangle, draw, text centered, rounded corners, node distance=2.2cm]
\tikzstyle{line} = [draw, -triangle 90]
\tikzstyle{dline} = [draw, dashed, -triangle 90]
%\tikzstyle{cloud} = [draw, ellipse,fill=red!20, node distance=3cm,minimum height=2em]
\section{Introduction}
\begin{frame}
\frametitle{Luminosity spectrum}
Why it's important to know it:
\begin{itemize}
\item cross section measurements: Higgs, etc.
\item mass measurements: slepton analysis, etc.
\end{itemize}
What we want to ``measure'': set of parameters that describes the luminosity
spectrum.
\begin{itemize}
\item need a correct model, based on assumptions and existing Monte Carlo
samples
\item need a framework/procedure for parameter estimation
\item need data 
\end{itemize}
\end{frame}

\section{Fitting methods}
\begin{frame}
\frametitle{Understanding the reweighting fit}
\begin{itemize}
  \item A \alert{distribution} of an observable $=$ \alert{``probability''} for
  an event to happen with a given observable value. Observable can be e.g.
  $\{EB1,EB2\}$ for a given event%, in fact 1 event
  %$=\{E1,E2,\theta_1,\theta_2,E_{ECAL1},E_{ECAL2}\}$)
  \item If said distribution is built from a set of parameters' values
  $[p]$, then probabilities can be computed from that set
  \item Changing $[p]\to [p]'\Rightarrow$ change of the probabilities. This
  change is accounted by the ratio $\frac{P(\{EB1,EB2\},
  [p]')}{P(\{EB1,EB2\},[p])}$. This weight is applied individually to every
  event.
  \item Bhabha events are not only represented by $\{EB1,EB2\}$ but by
  $\{EB1,EB2,\theta_1,\theta_2,E_{ECAL1},E_{ECAL2},etc.\}$ where the
  $\{\theta_1,\theta_2,E_{ECAL1},E_{ECAL2},etc.\}$ set are reconstructed
  (measured) observables
\end{itemize}
\end{frame}
\begin{frame}
\frametitle{Reweighting fit}
%Shown in fig.~\ref{fig:second}
%begin{figure}[h]
\begin{tikzpicture}[scale=.8,auto, remember picture]
\matrix [column sep=5mm,row sep=4mm,ampersand replacement=\&]{
%row1
\uncover<1>{\node [autoblock] (start) {\scriptsize Start};} \& 
~ \& 
~ \&
~ \\
%row2
\uncover<2->{\node [block, color=blue] (gen) {\scriptsize Generate Event with $[p]_0$: $E1$, $E2$, $P(E1,E2;[p]_0)$};} \&
\uncover<3-5>{\node [autoblock, node distance=2.7cm] (bhwide) {\scriptsize BHWide};} \&
\uncover<4-5>{\node [autoblock, node distance=2.5cm] (sim) {\scriptsize Simulation};} \&
\uncover<5->{\node [autoblock, node distance=2.7cm] (rec) {\scriptsize Reconstruction};} \\
%row3
\uncover<6->{\node [block, color=magenta] (minim) {\scriptsize Minimizer: $[p]_N \rightarrow [p]_{N+1}$};} \&
\uncover<7-13,15->{\node [block, text width=2.2cm, node distance=2.7cm] (compute) {\scriptsize Compute $P(E1,E2;[p]_{N+1})$ for all events};} \&
\uncover<8-13,16->{\node [autoblock, node distance=2.8cm] (weight) {\scriptsize $w = \frac{P(E1,E2;[p]_{N+1})}{P(E1,E2;[p]_{0})}$};} \&
\uncover<9-13,17->{\node [block, node distance=2.7cm] (weightrec) {\scriptsize Weight every event with its weight $w$};} \\
%row4
~ \& 
~ \& 
\uncover<11-13,19->{\node [decision, color=violet] (match) {\scriptsize
Minimum?};} \& \uncover<10-13,18->{\node [block, color=red] (compare) {\scriptsize Compare with data: $\chi^2$};}\\
%row5
~ \& 
~ \& 
\uncover<12-13,20->{\node [autoblock] (done) {\scriptsize Done!};} \& 
~ \\
};
\begin{scope}
\uncover<1>{\path [line] (start) -- (gen);}
\uncover<3-5>{\path [line] (gen) -- (bhwide);}
\uncover<4-5>{\path [line] (bhwide) -- (sim);}
\uncover<5>{\path [line] (sim) -- (rec);}
\uncover<6>{\path [line] (gen) -- (minim);}
\uncover<7-13,15->{\path [line] (minim) -- (compute);}
\uncover<8-13,16->{\path [line] (compute) -- (weight);}
\uncover<9-13,17->{\path [line] (weight) -- (weightrec);}
\uncover<10-13,18->{\path [line] (weightrec) -- (compare);}
\uncover<11-13,19->{\path [line] (compare) -- (match);}
\uncover<12-13,20->{\path [line] (match) -- node [near start] {\scriptsize yes} (done);}
\uncover<13,21->{\path [line] (match) -| node [very near start] {\scriptsize no} (minim);}
\uncover<9-13,17->{\path [dline] (rec) -- node [midway] {\scriptsize use} (weightrec); }
\end{scope}
\end{tikzpicture}
%\end{figure}
\end{frame}
\subsection{Status}
\begin{frame}
\frametitle{Current status}
Procedure for MC generation exists.\\
~\\
Generated MC samples and used Guinea Pig sample.\\
~\\
Several bugs found and fixed.
\end{frame}

\begin{frame}
\frametitle{Current status}
Fitting data before BHWide: compare the beam energies generated with our model
and the ones provided by GP.

\end{frame}

\end{document}
