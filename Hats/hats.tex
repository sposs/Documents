\documentclass[12pt]{article}
\title{Responsibilities}
\author{S. Poss}
\date{\today}
\begin{document}
\maketitle
\abstract{This document show the different responsibilities that have to be
taken to have a fully working system}

\section{Production Manager}
The Production Manager is in charge of producing data (duh!). That means
running the different applications in the right order (duh! again) to produce a
given amount of data. In details:
\begin{itemize}
  \item create productions
  \item monitor the status of productions with the web page: report to sys admin
  (see later) when problems arise.
  \item run the DataRecoveryAgent (until it is actually installed as an Agent
  in the server): recovers failed jobs that don't have uploaded the expected
  files
  \item prepare the production summary: the table for the twiki page
  \item Run productions in the right order: when generation has produced at
  least 1 file, definme the SIM production, then REC.
  \item stop productions that are done (to gain time essentially as they are
  not considered by the different agents), complete those that are really
  finished (removes all jobs from WMS, cleans sandbox files (log files)),
  and clean the productions are are invalidated (removes all, even data)
 \end{itemize}
 Amount of time per day: 1-2 h
 
\section{Data Manager}
The data manager is responsible for the data (haha). He is in charge of the
following:
\begin{itemize}
  \item monitoring of available disk space at the different SEs (lcg-infosites
  is useful)
  \item replication of data where needed: can be
  done using Replication transformation
  \item Monitor the replicas: where is what?
  \item staging/unstaging data (REC files in particular)
  \item replicating the software and making sure it's available
  everywhere
  \item Testing daily the SE access, and report to sys admins if something is
  wrong
  \item providing statisitics of usage
\end{itemize}
Amount of time per day: 1-2h

 \section{System Administrator}
 The system administrator is responsible for:
 \begin{itemize}
   \item monitoring the status of the vobox: disk, mem, cpu usage (SLS sensors
   can be used)
   \item monitoring the status of the agents and services (logging to vobox
   mandatory)
   \item reporting to DIRAC devs the bugs found
   \item Reporting to ILCDIRAC devs the issues
   \item ordering new voboxes
   \item request new space tokens for storage
   \item interact with computing elements when problems arise 
   \item Check global status of jobs: understand cause of failures in pilots
   and/or jobs
   \item Ban/unban sites
   \item Update DIRAC version to use when devs have done their work
   \item add new CE in the CS. 
 \end{itemize}
 Amount of work per day: 2h (but can be much more)
 
 \section{ILCDIRAC developer}
 \begin{itemize}
   \item Implement new functionality when users ask
   \item fix bugs
   \item test new DIRAC versions before putting them in production
   \item manage the repository
 \end{itemize}
 Amount of time per day: depends on the problems found
 
 \section{Responsible for generator}
 This role implies the following:
 \begin{itemize}
   \item installing whizard and PYTHIA (once only)
   \item adding new processes in whizard: add, recompile, test
   \item upload to GRID and register in DIRAC 
   \item Same for PYTHIA
   \item Prepare the grid files (result of whizard integration): run whizard on
   dedicated machine
 \end{itemize}
 Amount of time per day: 2-3 h.
\end{document}
