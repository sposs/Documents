\documentclass[10pt,table,dvipsnames]{beamer}
\usepackage{tikz}
\usepackage{mathptmx}
\usepackage[scaled=0.94]{helvet}
\usepackage[absolute,overlay]{textpos}

\usepackage{hyperref}

\author{S.~Poss}
\title{ILCDIRAC}
\subtitle{A grid solution for the ILC community}
\date{Jan 29, 2013}
\mode<all>
\TPGrid{50}{50}

\pgfdeclareimage[width=0.1\paperwidth]{cliclogo}{CLIClogo}
\newcommand{\ClicLogo}{%
\begin{textblock}{14}(45., 0.05)
 \href{http://lcd.web.cern.ch}{\pgfuseimage{cliclogo}}
\end{textblock}
}

\setbeamertemplate{footline}
{%
  \leavevmode%
  \hbox{%
  \begin{beamercolorbox}[wd=.222222\paperwidth,ht=2.25ex,dp=1ex,left]{title in 
head/foot}%
    \usebeamerfont{date in head/foot}\insertshortdate{}\hspace*{2em}
  \end{beamercolorbox}%
  \begin{beamercolorbox}[wd=.555555\paperwidth,ht=2.25ex,dp=1ex,center]{author 
in head/foot}%
    \usebeamerfont{author in head/foot}\insertshortauthor{}:
    \usebeamerfont{title in head/foot}\insertshorttitle
  \end{beamercolorbox}%
  \begin{beamercolorbox}[wd=.222222\paperwidth,ht=2.25ex,dp=1ex,right]{date in 
head/foot}%
    \insertframenumber{}/\inserttotalframenumber\hspace*{2ex}
  \end{beamercolorbox}}%
  %\vskip0pt%
  \ClicLogo
}

\beamertemplatenavigationsymbolsempty
\setbeamertemplate{blocks}[rectangle]
\setbeamersize{text margin left=1em,text margin right=1em}

%\setbeamertemplate{headline}[default]

\begin{document}
\begin{frame}
\titlepage
\end{frame}

\begin{frame}
\frametitle{Outline}
\tableofcontents
\end{frame}

\section{Introduction}\label{sec:intro}
\begin{frame}
  \frametitle{What is ILCDIRAC ?}
DIRAC client specific to the ILC community:
\begin{itemize}
\item DIRAC is a grid solution initially developed for LHCb
\item Extended for other communitites: Belle 2, BES, FERMI-LAT,
  CREATIS, ILC, etc.
\item Complete solution: Job Management, File cataloging, etc.
\end{itemize}
~\\
ILCDIRAC:
\begin{itemize}
\item Enables use of \alert{all ILC software} on the grid in a unified manner
\item Use of \alert{any} available resource (WLCG, OSG, local batch farms,
  etc.)
\item Has a failover mechanism: \alert{produced files cannot be lost}
\item Comes with a web portal that allows for most operations
\end{itemize}
\end{frame}

\section{Job Management}
\label{sec:jobman}

\subsection{Using the python API}
\label{sec:api}

\begin{frame}
  \frametitle{Job Management with the python API}
Idea: \alert{Users need to care what to run, not how.}
\begin{itemize}
\item All ILC applications are configured uniformly thanks to a unified interface
  \begin{itemize}
  \item Applications share logical properties like SteeringFile,
    Version, etc.
  \item Application specific things (Model, DST file name, etc.) have
    dedicated setters: they are only available to applications for
    which they make sense
  \end{itemize}
\item Data can be accessed from the catalog using meta data queries
  directly during job submission
  \begin{itemize}
  \item DIRAC File catalog has meta data information (and replica info)
  \end{itemize}
\item Steering files used for production are available directly to the
  users
  \begin{itemize}
  \item Only the file names are needed
  \end{itemize}
\item Output files are stored in a user configurable directory,
  without need of Storage Element specification (cloud-like)
\end{itemize}
\end{frame}

\subsection{Using the web portal}
\label{sec:web}

\begin{frame}
  \frametitle{Job Management with the Web portal}
  \begin{itemize}
  \item Possibility to submit small jobs through the portal directly
  \item Web interface to monitor the statuses, apply some operations
  (kill, reschedule, etc.)
\item Small sandboxes can be obtained from the portal
  \end{itemize}
\end{frame}

\section{File Catalog}
\label{sec:fc}
\begin{frame}
  \frametitle{File catalog}
\end{frame}

\section{Production Management}
\label{sec:ts}

\begin{frame}
  \frametitle{Production system}
Idea: Apply a set of operations to a set of files automatically
\begin{itemize}
\item Generation of a given channel at a given machine
\item Simulation, Reconstruction of a given channel
\item Transfer: replication from one site to another
\item Additing new tasks at the Generation steps automatically implies
  the creation of Simulation/Reconstruction of the new files: data
  driven process
\end{itemize}
~\\
All failing tasks that analyse files are reset: 99.9\% of the data
produced was simulated/reconstructed during the CLIC\_CDR production
\end{frame}

\section{Support}
\label{sec:support}

\begin{frame}
  \frametitle{Support}
JIRA, 
Community of admins
DIRAC developers
\end{frame}

\section{Conclusion}
\label{sec:conc}
\begin{frame}
  \frametitle{Conclusion}

\end{frame}
\end{document}


% Local Variables:
% TeX-PDF-mode: t
% End:
