\documentclass{beamer}
%\usepackage{beamerarticle}
\usepackage{tikz}

\usepackage{heppennames}
\usepackage{hepnicenames}
\usepackage{graphicx} 
\usepackage{multirow}
\usepackage{amsbsy,amsmath,amssymb}


%\usepackage{CJK}

\mode<presentation>
{
\usetheme{Singapore}
  \setbeamercovered{transparent}
   \setbeamertemplate{footline}[frame number] 
  \setbeamertemplate{navigation symbols}{ 
  \insertslidenavigationsymbol
  \insertframenavigationsymbol
  \insertsubsectionnavigationsymbol
  \insertsectionnavigationsymbol
  \insertdocnavigationsymbol
  \insertbackfindforwardnavigationsymbol
  \hskip 0.3cm
  %\insertframenumber / \inserttotalframenumber  % <<< frame #
  %\insertpagenumber / \insertpresentationendpage % <<< page #
} 
}

\usepackage[english]{babel}
\usepackage[latin1]{inputenc}

% font definitions, try \usepackage{ae} instead of the following
% three lines if you don't like this look
\usepackage{mathptmx}
\usepackage[scaled=.90]{helvet}
\usepackage{courier}


\usepackage[T1]{fontenc}

\title{Presentation}

%\subtitle{}

% - Use the \inst{?} command only if the authors have different
%   affiliation.
%\author{F.~Author\inst{1} \and S.~Another\inst{2}}
\author{St\'ephane Poss}
\institute[CERN]
{
Job reference: INPS12-8
}

% - Use the \inst command only if there are several affiliations.
% - Keep it simple, no one is interested in your street address.


\date{\today}


% This is only inserted into the PDF information catalog. Can be left
% out.
\subject{ILCDIRAC}



% If you have a file called "university-logo-filename.xxx", where xxx
% is a graphic format that can be processed by latex or pdflatex,
% resp., then you can add a logo as follows:

% \pgfdeclareimage[height=0.5cm]{university-logo}{university-logo-filename}
% \logo{\pgfuseimage{university-logo}}



% Delete this, if you do not want the table of contents to pop up at
% the beginning of each subsection:
\AtBeginSubsection[]
{
\begin{frame}<beamer>
\frametitle{Outline}
\tableofcontents[currentsection,currentsubsection]
\end{frame}
}

% If you wish to uncover everything in a step-wise fashion, uncomment
% the following command:

%\beamerdefaultoverlayspecification{<+->}
\begin{document}

\begin{frame}
\titlepage
\end{frame}

\begin{frame}
\frametitle{Curriculum}
\begin{enumerate}
  %\item 2001-2004, Physics Bachelor \\
  %  {\scriptsize (Universit\'e de la M\'editerrann\'ee, Marseille, France)}
  %\item 2004-2006, Physics Master \\
  %  {\scriptsize(Universit\'e de la M\'editerrann\'ee, Marseille, France)}
  \item 2006-2009, PhD. Flavour tagging in LHCb \\
    {\scriptsize(Universit\'e de la M\'editerrann\'ee, Marseille, France)}
  \item 2010-now, CERN fellowship: Linear Collider studies
\end{enumerate}
\end{frame}
 
%\part{Initial interest in High Energy Physics}
%\begin{frame}
%\partpage
%\end{frame}
% \begin{frame}
% \frametitle{Initial interest in High Energy Physics}
% First interest in HEP: at {\color{blue}15 years old}, high school:
% \begin{columns}
% \begin{column}[T]{3.5cm}
% \begin{center}
% \includegraphics[width=4cm]{briefhist}
% \end{center}
% \end{column}
% \begin{column}[T]{3cm}
% \begin{center}
% \includegraphics[width=2.7cm]{firstthree}
% \end{center}
% \end{column}
% \begin{column}[T]{5cm}
% \begin{center}
% \includegraphics[width=4.5cm]{firstcyclo}
% \end{center}
% \end{column}
% \end{columns}
% ~\\
% Wanted to work on accelerator physics, and make new discoveries\ldots\\
% \begin{center} 
% \alert{Decided to study high energy physics!}
% \end{center}
% \end{frame}
% 
% \begin{frame}
% \frametitle{Initial interest (Cont'd)}
% {\color{blue}Amazed by the cyclotron} studied in school:
% \begin{center}
% \includegraphics[width=6cm]{firstcyclo}
% \end{center}
% Wanted to work on accelerator physics, and make new discoveries\ldots\\
% \begin{center} 
% \alert{Decided to study high energy physics!}
% \end{center}
% \end{frame}

% \begin{frame}
% \frametitle{Initial interest (Cont'd)}
% University curriculum:
% \begin{itemize}
%   \item First 2 years in Aix-en-Provence, 
%   \item Then had to {\color{blue} move to Marseille} for the rest: Universit\'e
%   de la M\'editerrann\'ee
%   \item {\color{blue} Discovered CPPM} during first trip to University
%   \begin{center}
%   \includegraphics[width=2cm]{logocppm}
%   \end{center}
%   \item Knew I would do my \alert{PhD. thesis there} \pause
%   \item Managed to do that
% \end{itemize}
% \end{frame}
{
\usebackgroundtemplate{\includegraphics[width=\paperwidth]{gene-2008-002_01_t1}}

\part{2006-2009, PhD.: Flavour tagging in LHCb} 
\begin{frame}
\partpage
\end{frame}
\begin{frame}
\frametitle{Flavour tagging in LHCb}
Definition: determine the flavour (charge) of a b quark at its production.
\begin{center}
\includegraphics[width=11cm]{tagging.png}
\end{center}
Essential for many \alert{CP violation measurements}: $\sin(2\phi_1)$, $\beta_s$,
etc.
\end{frame}
}
% {
% \usebackgroundtemplate{\includegraphics[width=\paperwidth]{gene-2008-002_01_t2}}

% \begin{frame}
% \frametitle{From interships to the PhD.}
% \begin{itemize}
%   \item Started studying \alert{flavour tagging} between Bachelor and Master,
%   in the {\color{blue} LHCb group at CPPM, Marseille, Fr.}\\
%   ~\\
%   \item Internship at \alert{CERN in summer 2005}: Flavour tagging in
%     LHCb's event display, Panoramix\\
%   ~\\
%   \item Master's {\color{blue} internship}: Study of secondary
%   vertex reconstruction for flavour fagging in LHCb\\
%   ~\\
%   \item \alert{Accepted as PhD. student} in the LHCb group of CPPM
% \end{itemize}
% \end{frame}
% }
{
\usebackgroundtemplate{\includegraphics[width=\paperwidth]{gene-2008-002_01_t2}}

\begin{frame}
\frametitle{PhD.: Physics content}
Title: Calibration of the flavour tagging algorithm of the LHCb experiment by
the measurement of $\sin(2\phi_1)$\\
~\\
\begin{itemize}
  \item Selection of control channels: $\PBu\to\PJpsi\PKplus$ and
  $\PBd\to\PJpsi\PKstar^0$
  \item Measurement of the mistag fraction using $\PBd$ mixing property
  \item Measurement of $\sin(2\phi_1)$ in $\PBd\to\PJpsi\PKshort$ using
  previously measured mistag rate, systematics' studies
\end{itemize}
\begin{columns}[t]
\begin{column}[T]{4cm}
\begin{block}{$\PBu\to\PJpsi\PKplus$}
\includegraphics[width=4cm]{s7}
\end{block}
\end{column}
\begin{column}[T]{4cm}
\begin{block}{Mistag fraction}
\includegraphics[angle=90,width=4cm]{CombinedAsymFit}
\end{block}
\end{column}
\begin{column}[T]{4cm}
\begin{block}{$\sin(2\phi_1)$}
\includegraphics[angle=90,width=4cm]{AsymCPDC06AverageOmega}
\end{block}
\end{column}
\end{columns}
\end{frame}

 \begin{frame}
 \frametitle{PhD.: Using DIRAC in LHCb}
 Several {\color{blue}millions of events} to analyse + {\color{blue}thousands of
 toy MC studies}: used DIRAC a lot.
 \begin{center}
 \includegraphics[width=8cm]{sposs_user}
 \end{center}
 I wanted to \alert{be involved in the development of this tool}.
 \end{frame}
}


{
\usebackgroundtemplate{\includegraphics[width=10cm]{cern_logo_white}}
%%add CLIC logo: instead of ccern
\part{CERN Fellowship, 2010-now: Linear Collider detectors}
\begin{frame}
\partpage
\end{frame}
}
\begin{frame}
\frametitle{ILCDIRAC}
\begin{itemize}
  %\item Applied for \alert{fellowship at CERN}, emphasis on {\color{blue}DIRAC
  %in LHCb}: distributed analysis
  \item Developed a \alert{DIRAC client} for the ILC Virtual Organization (ILC and CLIC):\\ 
  \begin{itemize}  
    \item \alert{Mass production of Monte Carlo data} for the CLIC
    Conceptual Design Report (CDR): {\color{blue}benchmark of 2 detector
    concepts}\\~\\
    %\item Document {\color{blue} finished in 2011}
    %\item DIRAC was proven by LHCb to be \alert{efficient}
    \item Used by Linear Collider community
    %\item DIRAC team wanted to show it could be used outside LHCb
  \end{itemize}
\end{itemize}
\begin{columns}
  \column{0.5\textwidth}
\centering
  \includegraphics[width=\textwidth]{Detectors.png}
  \column{0.5\textwidth}
\centering
  \includegraphics[width=0.7\textwidth]{CDR_front}
\end{columns}
%add front page of CDR
\end{frame}

% \begin{frame}
% %group with previous
% \frametitle{Linear Collider detectors}
% \begin{columns}
% \begin{column}{6cm}
% \includegraphics[width=6cm]{Detectors.png}
% \end{column}
% \begin{column}{5cm}
% ILC and CLIC are future Linear Colliders\\
% ~\\
% Detectors' design similar to CMS\\
% ~\\
% Main difference between concepts: {\color{blue}the tracking system}
% \begin{itemize}
%   \item ILD uses a TPC
%   \item SiD uses silicon layers
% \end{itemize}
% \end{column}
% \end{columns}
% \end{frame}

{
\usebackgroundtemplate{\includegraphics[width=\paperwidth]{DiracVisual}} 
 
\begin{frame}
\frametitle{The ILCDIRAC client}
ILCDIRAC: DIRAC client dedicated to the {\color{blue}linear collider
community}:
\begin{itemize}
  \item ILC and CLIC share the same virtual organisation  
    \begin{itemize}
    \item CLIC detectors studies use ILC detectors software
    \end{itemize}
% ILC made software availalble to us
  \item \alert{Convenient interface} to handle 12 ILC applications\\
    (Whizard, Mokka, Marlin, SLIC, LCSIM, etc.)
  \begin{itemize}
  \item Integration of software frameworks
  \item Distribution of packages on the GRID
  \item Users only care about \alert{what to do, not how}.
  \end{itemize}
\item Written in {\color{blue}PYTHON} to follow DIRAC framework
\item \alert{Documentation/Tutorials} 
\item Now \alert{used by the ILC SID community} for their Detailed Baseline
  Design document mass production
  \begin{itemize}
  \item  ILC studies use CLIC software
  \end{itemize}
%ILC uses now CLIC software
\end{itemize}
%CLIC is using ILC software
\end{frame}

\begin{frame}
  \frametitle{Usage}
ILCDIRAC client was used for:
\begin{itemize}
  \item Mass Monte Carlo \alert{production}: CLIC CDR and SID DBD\\~\\
  \item User jobs: \alert{analysis} for ILC and CLIC\\~\\
  \item \alert{Data management: File Catalog} was essential for those activities\\~\\
\end{itemize}
~\\
More than {\color{blue}90 million events (ILC and CLIC) processed in
  2.5 year}, \alert{$\sim100$ channels}
\\%event number instead
%%add plot
~\\
Users not only at CERN, but also LAL (FR.), MPI (DE.), VINCA (R.S.),
SLAC (U.S.A), etc.
\end{frame}
\begin{frame}
  \frametitle{Performance of the system}
\centering
\begin{tikzpicture}
  \node {\includegraphics[width=0.8\textwidth]{CumulJobsPerSite}};
  \draw [<->, color=blue](-2.7,1.4) -- node [midway,above]
  {{\scriptsize CLIC CDR}}(0.,1.4);
  \draw [<->, color=blue](2.,1.6) -- node [midway,above] {{\scriptsize CLIC CDR}}(3.,1.6);
  \draw [<->, color=blue](3.5,2.5) -- node [midway,above] {{\scriptsize SID DBD}}(4.2,2.5);
\end{tikzpicture}\\
Also: \alert{6.3 million files} in the File Catalog, 450TB on
different distributed storage elements
\end{frame}

\begin{frame}
\frametitle{My responsibilities}
ILCDIRAC management:
\begin{itemize}
  \item Development of \alert{new features}
  \item {\color{blue}Installation and setup} of services (Workload Management, Data Management, Production)
  \item \alert{Monitoring} of system status: VOBOX and GRID resources
  \item Realisation of \alert{documentation}: tutorial slides, online code
  documentation
  \item Interaction with DIRAC main developers
  \item Student supervision
\end{itemize}
~\\
Mass Production:
\begin{itemize}
   \item \alert{Production manager}: definition of new productions, monitor
   statuses, produce statistics
   \item {\color{blue}Data manager}: make sure the data is where it's supposed to
   be, replicate when needed, check availability of resources
\end{itemize}

\end{frame}
}

% {
% \usebackgroundtemplate{\includegraphics[width=\paperwidth]{prod_mon2}}

% \begin{frame} 
% \frametitle{My current activities (Cont'd)}
% \end{frame}
% }
% {
% \usebackgroundtemplate{\includegraphics[width=\paperwidth]{tt-mass2}} 

% \begin{frame}
% \frametitle{Current activities (Cont'd)}
% Convener of the $\Ptop\APtop$ at $\sqrt{s}=$500~GeV analysis:\\
% \begin{itemize}
%   \item One of the 6 benchmark channels for the CLIC detectors~\\ ~\\
%   \item Interesting as a comparison point with ILC detectors
% \end{itemize}
% \end{frame}
%  }
 %  {
 % \usebackgroundtemplate{\includegraphics[width=\paperwidth]{predator}} 
 
 %  \begin{frame}
 %  \frametitle{Student supervision}
 %  P. Majewski (Master student): 
 %  \begin{itemize}
 %    \item Initial developments of ILCDIRAC
 %  \end{itemize}
 %  ~\\
 %  C.~B.~Lam (Bachelor student):
 % \begin{itemize}
 %   \item Quality control of Production Data
 %   \item Jet energy resolution effects on top quark mass measurements
 %   \item Review of ILCDIRAC user interface
 % \end{itemize}
 % ~\\
 % E. Hidle (Master student):
 % \begin{itemize}
 %   \item GRID job running time predictions, using Case Based Reasoning in the
 %   ILCDIRAC context
 % \end{itemize}
 %  \end{frame}
 % }

%{
%\usebackgroundtemplate{\includegraphics[width=\paperwidth]{LumiSpectrum}}
  \begin{frame}
    \frametitle{Additional activities}
    Measurement of the \alert{luminosity spectrum} (at CLIC 3TeV):
    \begin{itemize}
      \item \alert{Beam-beam interactions}: beamstrahlung, beam-energy spread
      \item Needed for {\color{blue}precise cross section} measurements
      \item Provide spectra to users accounting for statistical and %%rephrase
        systematic errors
      \item Implement models using \alert{Object Oriented principles} (C++)
    \end{itemize}
%FIXME
%\hskip 0.3\textwidth
\centering 
\includegraphics[width=0.5\textwidth]{Refh2}\\
~\\
\end{frame}
%}
\part{Proposed contributions to Belle II}
{
\usebackgroundtemplate{\includegraphics[width=\paperwidth]{belle2-logo_transp}}
\begin{frame}
\partpage
\end{frame}
\begin{frame}
  \frametitle{Ideas: Computing}
Development:
\begin{itemize}
\item \alert{User interface}
\item Data Catalog: \alert{DIRAC File Catalog}
\item Data management: {\color{blue}``popularity''}
\item DIRAC mass {\color{blue}production system}
\end{itemize}
~\\
Operations:
\begin{itemize}
\item Interface with Computing Elements: Amazon EC2, GRID (PNNL)
\item Service monitoring, machine management
\item Data replication, production management
\item Training
\end{itemize}
\end{frame}

\begin{frame}
  \frametitle{Ideas: Physics}
Data analysis:
  \begin{itemize}
  \item \alert{Flavour tagging} optimization and calibration
  \item Background rejection
  \item {\color{blue}Luminosity spectrum} measurement
  \item $\PB\to \tau \nu$: hints to New Physics
  \item \ldots
  \end{itemize}
\end{frame}
}
\part{Future: ILC}
{
\usebackgroundtemplate{\includegraphics[width=\paperwidth]{logo_ilc_white_bg}}
\begin{frame}
\partpage
\end{frame}
\begin{frame}
  \frametitle{The future}
 ILC detectors will need computing infrastructures:
 \begin{itemize}
  \item High event rate
  \item High event density
  \item International collaboration: data distribution
 \end{itemize}
~\\
Already familiar with:
\begin{itemize}
\item Detector frameworks
\item Physics conditions
\item Software environment
\end{itemize}
\end{frame}
}
 % \begin{frame}
 %   \vskip 5cm 
 %   \centering
 %   %toto
 %   % \begin{CJK}{utf8}{min}
 %   \begin{CJK}{UTF8}{min}
 %     ���꤬�Ȥ�
 %   \end{CJK}
 %\end{frame}
 \part{Summary in relation to announced position}
 \begin{frame}
 \partpage
 \end{frame}
 \begin{frame}
 \frametitle{Summary} 
 \begin{itemize}
 \item \alert{Computing infrastructure} development
   and operation:
   \begin{itemize}
   \item {\color{blue}DIRAC for Belle 2}
   \item {\color{blue}Production} activities
   \item {\color{blue}Simulation and Reconstruction} software frameworks
   %\item Interesting prospects!
   \end{itemize}
   ~\\
 \item \alert{Data} taking and analysis:
   \begin{itemize}
   \item {\color{blue}CP violation} still has a lot to tell
   \item {\color{blue}Rare decays}: provide hints on New Physics
   %\item Very exciting subject! 
   \end{itemize}
   ~\\
 \item Open to any future challenges!
 \end{itemize}
%~\\
%\hspace{0.6\textwidth} \scriptsize{Japan is a fascinating country!}  
 \end{frame}

\appendix
\begin{frame}
  \frametitle{Backups}
\end{frame}
{
\usebackgroundtemplate{\includegraphics[width=\paperwidth]{process_ex_1}}

\begin{frame}
\frametitle{Other activities}
Physics generation:\\
\begin{itemize}
  \item Setup framework for {\color{blue} convenient physics generation}: 2
  generators, WHIZARD and PYTHIA~\\ ~\\
  \item Implement channels in the 2 generators used, perform tests~\\ ~\\
  \item \alert{Interface to ILCDIRAC}
\end{itemize}
CERN representative of the working group dedicated to {\color{blue}common
generator tools for the Linear Colliders}.
\end{frame}
}

\end{document}
% Local Variables:
% TeX-PDF-mode: t
% coding: euc-japan
% End:
