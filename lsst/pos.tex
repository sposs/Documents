\documentclass[10pt,table,dvipsnames]{beamer}
\usepackage{tikz}
\usepackage{mathptmx}
\usepackage[scaled=0.94]{helvet}
\usepackage[absolute,overlay]{textpos}
\usepackage{hyperref}
\usepackage{listings}
\lstloadlanguages{python}

\author{S.~Poss}
\title{DIRAC Transformation System}
\subtitle{un coups d'oeil}
\date{11 avril 2013}
\institute{CERN, LAPP}
\newcommand{\interstitial}[1]{\begin{frame}\begin{block}{}\centering\Huge{#1}\end{block} \end{frame}}
\newcommand{\backupslides}{\interstitial{Backup Slides}}

\mode<all>
\TPGrid{50}{50}

\pgfdeclareimage[width=0.1\paperwidth]{cliclogo}{CLIClogo}
\newcommand{\ClicLogo}{%
\begin{textblock}{14}(45., 0.05)
 \href{http://lcd.web.cern.ch}{\pgfuseimage{cliclogo}}
\end{textblock}
}

\setbeamertemplate{footline}
{%
  \leavevmode%
  \hbox{%
  \begin{beamercolorbox}[wd=.222222\paperwidth,ht=2.25ex,dp=1ex,left]{title in 
head/foot}%
    \usebeamerfont{date in head/foot}\insertshortdate{}\hspace*{2em}
  \end{beamercolorbox}%
  \begin{beamercolorbox}[wd=.555555\paperwidth,ht=2.25ex,dp=1ex,center]{author 
in head/foot}%
    \usebeamerfont{author in head/foot}\insertshortauthor{}:
    \usebeamerfont{title in head/foot}\insertshorttitle
  \end{beamercolorbox}%
  \begin{beamercolorbox}[wd=.222222\paperwidth,ht=2.25ex,dp=1ex,right]{date in 
head/foot}%
    \insertframenumber{}/\inserttotalframenumber\hspace*{2ex}
  \end{beamercolorbox}}%
  %\vskip0pt%
  %\ClicLogo
}

\beamertemplatenavigationsymbolsempty
\setbeamertemplate{blocks}[rectangle]
\setbeamersize{text margin left=1em,text margin right=1em}

%\setbeamertemplate{headline}[default]

\begin{document}
\renewcommand{\inserttotalframenumber}{\ref{lastframe}}

\begin{frame}
\titlepage
\end{frame}

%\begin{frame}
%\frametitle{Outline}
%\tableofcontents
%\end{frame}

\begin{frame}
  \frametitle{Transformation System}

  \begin{description}
    \item[Transformation:] Op\'eration sur un lot de fichiers~: cr\'eation et/ou modification \\
~\\
    \item[T\^ache:] Element d'une transformation: associe un/plusieurs fichier(s) \`a
      une op\'eration
  \end{description}
\end{frame}

\begin{frame}
  \frametitle{Objectif}
\begin{enumerate}
\item Obtenir un syst\`eme permettant l'ex\'ecution de t\^aches similaires sur un
lot de donn\'ees\\~\\
\item Automatisation de la proc\'edure\\~\\
\item Encha\^inement simple de t\^aches\\~\\
\item Limiter les actions manuelles\\~\\
\item Ne pas re-traiter les fichiers d\'ej\`a utilis\'es
\end{enumerate}
\end{frame}

\begin{frame}
  \frametitle{Comment ?}
Le Transformation System permet ceci:
\begin{itemize}
\item Avec un ``template'' de job (workflow)
\item avec un lot de fichier (input des jobs, peut \^etre un chemin ou autre)
\end{itemize}
~\\
Il cr\'ee les t\^aches qui conviennent:
\begin{description}
\item[TransformationAgent] cr\'e\'e les t\^aches: associe un fichier \`a une t\^ache
  (plusieurs r\`egles possible, voir plus loin)
\item[Workflow/RequestTaskAgent] envoie les t\^aches au syst\`eme correspondant: WMS pour les
  jobs, Request Manager pour les requ\^etes (r\'eplication par exemple)
\item[DataRecoveryAgent] recup\`ere les jobs failed et change le status des fichiers
  en input pour que l'agent 1 recr\'e\'e une/plusieurs t\^ache(s)
  (n'existe pas par defaut)
\item[InputDataAgent] met \`a jour la liste d'input par transformation suivant
  des requ\^etes de meta data dans le catalogue (data-driven procedure).
\item[TransformationCleaningAgent] nettoie les transformations lorsque
  necessaire: nettoie les tables, retire les fichiers si necessaire,
  nettoie les catalogues
\end{description}
\end{frame}

\begin{frame}
  \frametitle{Comment ?}
Utilise le status des fichiers:
\begin{description}
\item[Unused:] Le fichier n'est pas encore utilis\'e
\item[Assigned:] le fichier est associ\'e \`a une t\^ache: il n'est
  pas consid\'er\'e lors de la cr\'eation d'une nouvelle t\^ache
\item[Processed:] Le fichier a \'et\'e trait\'e: g\'en\'eralement, ce
  satus est mis \`a jour \`a la fin d'un job (en appelant le
  \emph{FileReport}). Si le job est 'failed' et la raison comprise alors le fichier peut
  \^etre marqu\'e comme 'Unused'.
\item[MaxReset:] Lorsqu'un fichier \'echoue trop souvent, il est
  marqu\'e de mani\`ere \`a ne pas recr\'eer de nouvelle t\^ache. Peut
  \^etre ignor\'e par le DataRecoveryAgent s'il est utilis\'e.
\end{description}
\end{frame}

\begin{frame}
  \frametitle{D\'efinition du template}
  \begin{itemize}
  \item Techniquement, un template peut \^etre n'importe quoi: c'est un BLOB
dans la TransformationDB. \\
~\\
  \item Ajustement des agents n\'ecessaire si ce n'est pas un DIRAC
    Workflow: reimpl\'ementation du WorkflowTaskAgent
  \end{itemize}

~\\

(\emph{t.setBody()})
\end{frame}

\begin{frame}
  \frametitle{R\`egles de cr\'eation des t\^aches (TransformationPlugin)}
Tous les fichiers disponibles et non associ\'es \`a une t\^ache sont utilis\'es
pour cr\'eer de nouvelles t\^aches en utilisant les r\`egles suivantes:
  \begin{description}
  \item[Standard] Groupe les fichiers par site, avec la taille du
    groupe choisie lors de la cr\'eation de la transformation
    (\emph{t.setGroupSize()}).
  \item[Broadcast] Utilis\'e pour les replications: assigne les
    destinations des fichiers (\emph{t.setTargetSE()})
  \item[ByShare] S'assure que tout les sites ont une
    utilisation \'equivalente
  \item[BySize] Groupe les fichiers d'input de mani\`ere \`a avoir une
    utilisation du disque \'equivalente (pas souvent utilis\'e)
  \end{description}
~\\
Possibilite d'\'etendre ces r\`egles simplement (subclass). \\

~\\

(\emph{t.setPlugin()})
\end{frame}

\begin{frame} 
  \frametitle{Exemple}
\lstinputlisting[language=python,basicstyle=\footnotesize]{SimpleJob.py} 
\end{frame}

\begin{frame}
  \frametitle{Monitoring}
Depuis le portail DIRAC (Matvey): 
\begin{itemize}
\item Suivi de l'\'evolution de chaque transformation
\item Op\'erations: start, stop, clean, complete (, flush) 
\item Obtension des propri\'et\'es de chaque transformation
\item Possibilit\'e d'inspecter chaque job 
\item Extension de certaines transformations
\end{itemize}
\end{frame}

\begin{frame}
\frametitle{Conclusion}
Si un ensemble de t\^aches identiques doivent \^etre appliqu\'ees sur
un lot de donn\'ees, alors le TS est le bon choix

~\\

L'enchainement des t\^aches est possible en utilisant l'aspect
data-driven

~\\

L'extension des fonctionnalit\'es est possible en cr\'eant de nouvelles classes/services/agents
\label{lastframe}
\end{frame}

\backupslides 


\end{document}
% Local Variables:
% TeX-PDF-mode: t
% End:
