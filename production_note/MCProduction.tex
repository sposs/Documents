\documentclass{article}

\author{Stephane Poss}
\title{Monte Carlo production for the CLIC CDR}
\date{\today}

\begin{document}
\maketitle
\abstract{This note presents the Monte Carlo mass production methodology used for the CLIC Conceptual design report. 
First the framework used, DIRAC, is briefly presented. Then we go through the different steps of the production: 
Generation, Simulation, and Reconstruction, with and without overlay. Finally, we present the overall resources needed,
in terms of number of job, equivalent CPU time, and disk space.}
\section{Introduction}
As the CDR is being completed, it is useful to have a review of the Monte Carlo production process. The main aim is to 
have in one location the different parameters used througout the production. We also refer to external documents when 
relevant. 

After a brief overview of the framework used for the production, namely DIRAC, we present the individual steps
of the production, with their parameters. Finally, the resources used are presented: total number of jobs, equivalent
CPU time consumed, average memory used, and final disk space required.

\section{Framework and tools: DIRAC}

\section{Generation}

\section{Simulation}

\section{Reconstruction}
\subsection{Without overlay}

\subsection{With overlay}

\section{Resources used}

\section{Prospects}
\section{Conclusion}

\end{document}
