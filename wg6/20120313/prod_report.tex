\documentclass{beamer}
%\usepackage{beamerarticle}
%%\usepackage{heppennames}
%%\usepackage{hepnicenames}
\usepackage{graphicx} 
\usepackage{multirow}
\usepackage{amsbsy,amsmath,amssymb}

\mode<presentation>
{
\usetheme{Singapore}
  \setbeamercovered{transparent}
   \setbeamertemplate{footline}[frame number] 
  \setbeamertemplate{navigation symbols}{ 
  \insertslidenavigationsymbol
  \insertframenavigationsymbol
  \insertsubsectionnavigationsymbol
  \insertsectionnavigationsymbol
  \insertdocnavigationsymbol
  \insertbackfindforwardnavigationsymbol
  \hskip 0.3cm
  %\insertframenumber / \inserttotalframenumber  % <<< frame #
  %\insertpagenumber / \insertpresentationendpage % <<< page #
} 
}

\usepackage[english]{babel}
\usepackage[latin1]{inputenc}

% font definitions, try \usepackage{ae} instead of the following
% three lines if you don't like this look
\usepackage{mathptmx}
\usepackage[scaled=.90]{helvet}
\usepackage{courier}


\usepackage[T1]{fontenc}

\usepackage{listings}
\lstloadlanguages{bash}

\title{Production status}

\subtitle{Where we are}

% - Use the \inst{?} command only if the authors have different
%   affiliation.
%\author{F.~Author\inst{1} \and S.~Another\inst{2}}
\author{S.~Poss}

% - Use the \inst command only if there are several affiliations.
% - Keep it simple, no one is interested in your street address.
\institute[CERN]
{%
CERN, Switzerland
}

\date{\today}


% This is only inserted into the PDF information catalog. Can be left
% out.
\subject{ILCDIRAC}



% If you have a file called "university-logo-filename.xxx", where xxx
% is a graphic format that can be processed by latex or pdflatex,
% resp., then you can add a logo as follows:

% \pgfdeclareimage[height=0.5cm]{university-logo}{university-logo-filename}
% \logo{\pgfuseimage{university-logo}}



% Delete this, if you do not want the table of contents to pop up at
% the beginning of each subsection:
\AtBeginSubsection[]
{
\begin{frame}<beamer>
\frametitle{Outline}
\tableofcontents[currentsection,currentsubsection]
\end{frame}
}

% If you wish to uncover everything in a step-wise fashion, uncomment
% the following command:

%\beamerdefaultoverlayspecification{<+->}

\begin{document}

\begin{frame}
\titlepage
\end{frame}

\begin{frame}
\frametitle{Introduction}
\begin{itemize}
  \item New production round is running
  \item Overlay problems (again\ldots)
\end{itemize}
\end{frame}

\begin{frame}
\frametitle{Changes wrt last round}
Between 100 and 1000 events per file (job). 

Limit to 4 files to obtain for the overlay.
\end{frame}

\begin{frame}
\frametitle{Current status}
\begin{itemize}
  \item H recoil: all done (gen+sim+rec overlay) but $\mu \mu f \bar{f}$ 
  \item stau: all done (gen+sim) but $ee\tau\tau$
  \item gauginos: all done (gen+sim) but $qqqq\nu\nu$ (missing grid files)
  \item triple Higgs, top: nothing yet, waiting for statistics needed
\end{itemize}
\end{frame}

\begin{frame}
\frametitle{Issues}
\begin{itemize}
  \item Simulation of events with electrons much slower than I expected (i.e.
  high failure rate), but normal as level of detail in ECAL much higher
  \item Overlay: as before, access to files problematic. Example: for 992
  files ($ee f \bar{f}$), 3600 jobs needed. All jobs at CERN fail\ldots
  \item Expect to be a problem for 1.4TeV: will have to go very slowly.
\end{itemize}
\end{frame}

\begin{frame}
\frametitle{Conclusion}
Look at https://twiki.cern.ch/twiki/bin/view/CLIC/GeneratorFiles (bottom) for
current status of generation. Simulation and Reconstruction statistics will be
prepared.

\end{frame}
\end{document}