\documentclass{beamer}
%\usepackage{beamerarticle}
%%\usepackage{heppennames}
%%\usepackage{hepnicenames}
\usepackage{graphicx} 
\usepackage{multirow}
\usepackage{amsbsy,amsmath,amssymb}

\mode<presentation>
{
\usetheme{Singapore}
  \setbeamercovered{transparent}
   \setbeamertemplate{footline}[frame number] 
  \setbeamertemplate{navigation symbols}{ 
  \insertslidenavigationsymbol
  \insertframenavigationsymbol
  \insertsubsectionnavigationsymbol
  \insertsectionnavigationsymbol
  \insertdocnavigationsymbol
  \insertbackfindforwardnavigationsymbol
  \hskip 0.3cm
  %\insertframenumber / \inserttotalframenumber  % <<< frame #
  %\insertpagenumber / \insertpresentationendpage % <<< page #
} 
}

\usepackage[english]{babel}
\usepackage[latin1]{inputenc}

% font definitions, try \usepackage{ae} instead of the following
% three lines if you don't like this look
\usepackage{mathptmx}
\usepackage[scaled=.90]{helvet}
\usepackage{courier}


\usepackage[T1]{fontenc}

\usepackage{listings}
\lstloadlanguages{bash}

\title{Production status}

\subtitle{Changes for next round}

% - Use the \inst{?} command only if the authors have different
%   affiliation.
%\author{F.~Author\inst{1} \and S.~Another\inst{2}}
\author{S.~Poss}

% - Use the \inst command only if there are several affiliations.
% - Keep it simple, no one is interested in your street address.
\institute[CERN]
{%
CERN, Switzerland
}

\date{\today}


% This is only inserted into the PDF information catalog. Can be left
% out.
\subject{ILCDIRAC}



% If you have a file called "university-logo-filename.xxx", where xxx
% is a graphic format that can be processed by latex or pdflatex,
% resp., then you can add a logo as follows:

% \pgfdeclareimage[height=0.5cm]{university-logo}{university-logo-filename}
% \logo{\pgfuseimage{university-logo}}



% Delete this, if you do not want the table of contents to pop up at
% the beginning of each subsection:
\AtBeginSubsection[]
{
\begin{frame}<beamer>
\frametitle{Outline}
\tableofcontents[currentsection,currentsubsection]
\end{frame}
}

% If you wish to uncover everything in a step-wise fashion, uncomment
% the following command:

%\beamerdefaultoverlayspecification{<+->}

\begin{document}

\begin{frame}
\titlepage
\end{frame}

\begin{frame}
\frametitle{Introduction}
\begin{itemize}
  \item New production round is coming
  \item Need to fix some of the problems faced
  \item Not only for CLIC vol3, but also for DBD
\end{itemize}
\end{frame}

\begin{frame}
\frametitle{Problems in previous round}
\begin{itemize}
  \item Lots of bookkeeping: manual definition of simulation then reconstruction
  productions
  \item Metadata registration: done for every job, very inefficient for the
  FileCatalog
  \item Some metadata missing
  \item Very few events per job: time constrain
\end{itemize}
\end{frame}

\begin{frame}
\frametitle{Changes for next round}
\begin{itemize}
  \item Use new DIRAC interface: flexible configuration of jobs
  \item Metadata registration done when production is created: picks the needed
  info from input
  \item Add non-searchable metadata: software used, detector models, steering
  files, whatever
  \item Number of events set for every file
\end{itemize}
\end{frame}

\begin{frame}
\frametitle{Current status}
\begin{itemize}
  \item Lumi spectrum produced: checks ongoing
  \item Tested system: Standard Model and SUSY, Whizard+Mokka
  \item 1000 events possible, not for all event types though
  \item Simulated $\gamma\gamma \to had$ with Mokka: no problem
  \item Same with SLIC: crash. Reason: the samples contain only final state
  particles, some outside world volume. Mokka brings back all particles to IP.
  Both approach are wrong: Daniel will regenerate the sample. 
\end{itemize}
\end{frame}
\begin{frame}
\frametitle{Finally}
\begin{itemize}
  \item Can think about defining ALL productions from the beginning: Gen+Sim+Rec
  \item Need to define all steering files and parameters, and cast them in
  stone. E.g. use QGSP\_BERT\_HP?
\end{itemize}
\alert{What about overlay?}
\end{frame}
\end{document}