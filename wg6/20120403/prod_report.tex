\documentclass{beamer}
%\usepackage{beamerarticle}
%%\usepackage{heppennames}
%%\usepackage{hepnicenames}
\usepackage{graphicx} 
\usepackage{multirow}
\usepackage{amsbsy,amsmath,amssymb}
\usepackage{booktabs}

\mode<presentation>
{
\usetheme{Singapore}
  \setbeamercovered{transparent}
   \setbeamertemplate{footline}[frame number] 
  \setbeamertemplate{navigation symbols}{ 
  \insertslidenavigationsymbol
  \insertframenavigationsymbol
  \insertsubsectionnavigationsymbol
  \insertsectionnavigationsymbol
  \insertdocnavigationsymbol
  \insertbackfindforwardnavigationsymbol
  \hskip 0.3cm
  %\insertframenumber / \inserttotalframenumber  % <<< frame #
  %\insertpagenumber / \insertpresentationendpage % <<< page #
} 
}

\usepackage[english]{babel}
\usepackage[latin1]{inputenc}

% font definitions, try \usepackage{ae} instead of the following
% three lines if you don't like this look
\usepackage{mathptmx}
\usepackage[scaled=.90]{helvet}
\usepackage{courier}


\usepackage[T1]{fontenc}

\usepackage{listings}
\lstloadlanguages{bash}

\title{Production status}

\subtitle{Where we are}

% - Use the \inst{?} command only if the authors have different
%   affiliation.
%\author{F.~Author\inst{1} \and S.~Another\inst{2}}
\author{S.~Poss}

% - Use the \inst command only if there are several affiliations.
% - Keep it simple, no one is interested in your street address.
\institute[CERN]
{%
CERN, Switzerland
}

\date{\today}


% This is only inserted into the PDF information catalog. Can be left
% out.
\subject{ILCDIRAC}



% If you have a file called "university-logo-filename.xxx", where xxx
% is a graphic format that can be processed by latex or pdflatex,
% resp., then you can add a logo as follows:

% \pgfdeclareimage[height=0.5cm]{university-logo}{university-logo-filename}
% \logo{\pgfuseimage{university-logo}}



% Delete this, if you do not want the table of contents to pop up at
% the beginning of each subsection:
\AtBeginSubsection[]
{
\begin{frame}<beamer>
\frametitle{Outline}
\tableofcontents[currentsection,currentsubsection]
\end{frame}
}

% If you wish to uncover everything in a step-wise fashion, uncomment
% the following command:

%\beamerdefaultoverlayspecification{<+->}

\begin{document}

\begin{frame}
\titlepage
\end{frame}

\begin{frame}
\frametitle{Introduction}
\begin{itemize}
  \item New production round is running
  \item Issues with beam spectrum
\end{itemize}
\end{frame}

\begin{frame}
\frametitle{Status}
\begin{itemize}
  \item Samples for Higgs recoil analysis done, validation done by John,
  analysis ongoing.
  \item Stau samples done (even the templates), validation done by Astrid.
  Analysis ongoing.
  \item Gauginos: Signal samples done, most background done. Only qqqqnunu
  missing. Templates submitted. Validation done by Philipp, analysis ongoing.
  \item triple H 1.4Tev: signal submitted and running, as all backgrounds. Need
  validation of the samples.
  \item trplie H 3TeV: need the final grid files for the backgrounds. Need
  validation of the samples.
  \item Sleptons: produced by JJ Blaising, running. 
  \item ttbar: Pythia program revived, need input: 350GeV spectrum and
  gg$\to$had parameters (can probably use the same events as 500GeV).
\end{itemize}
\end{frame}

\begin{frame}
\frametitle{Issues with 1.4TeV beam spectrum}
\begin{itemize}
  \item When looking at the templates, Philipp noticed an inconsitency in the
  cross sections
  \item tracked down to the fact that most of those used an ``old'' version of
  the spectrum
  \item The problem: some sites allow anyone to write in the SharedArea, and the
  software is not overwritten if present (unless explicitely requested).
  \item Created the list of files affected by the problem by parsing the log
  files.
  \item Next slides give this list of affected productions and the treatment
  applied.
\end{itemize}
\end{frame}

\begin{frame}
\frametitle{Productions affected and their treatment}
\begin{center}
{\scriptsize 
\begin{tabular}{ccccc}
    \toprule
ProdID & Channel & Nb of faulty files & \% affected & treatment \\
\midrule
788 & e3e3nn & 2 & 1 & will delete and extend existing\\
790 & e1e1e3e3\_o & 1  & $<1$ & delete, no need to reproduce\\
794 & aa\_e3e3\_o  & 1 &$<<1$& delete, no need to reproduce\\
800& aa\_e3e3nn & 12 & $<1$ & delete and extend existing\\
805& neu2neu2 & 1 & 1 & delete, no need to reproduce\\
807& ch1ch1\_nunu & 1 & 4 & delete and extend existing\\
811& hhnunu & 1 & 4 & delete, no need to reproduce\\
813& hqqnunu & 6 & 14 & delete and extend existing \\
831& e3e3 & 226 & 50 & will check with astrid \\
853& hh\_nunu & 79 & 80 & restart from scratch\\
860& qqqq\_nunu & 612 &~90& restart from scratch\\
888& qqqq & 77 & 70 & restart from scratch\\
890& h\_n1n1 & 1 & 100 & delete and extend existing\\
892& hh\_nunu & 1 & 50 & delete and extend existing\\
    \bottomrule
\end{tabular}
}
\end{center}
\end{frame}

\begin{frame}
\frametitle{Productions affected and their treatment}
All templates for the gauginos are concerned, they are all removed and restarted
from scratch.
\begin{center}
{\scriptsize 
\begin{tabular}{ccccc}
    \toprule
ProdID & Channel & Nb of faulty files & \% affected & treatment \\
\midrule
987 & stau1stau1 & 13  & 50 &  delete and extend existing\\
990 & stau1stau1 & 19  & 60 &  delete and extend existing\\
993 & stau1stau1 & 8 & 40 &  delete and extend existing\\
996 & stau1stau1 & 13 & 50 &  delete and extend existing\\
999 & stau1stau1 & 14  & 50 &  delete and extend existing\\
1002 & stau1stau1 & 15 & 50 &  delete and extend existing\\
1013 & qqqq\_lnu & 3  & 100 & delete and extend existing\\
     \bottomrule
\end{tabular}
}
\end{center}

Beam spectra used now cannot 

\end{frame}

\begin{frame}
\frametitle{Conclusion}
\begin{itemize}
  \item Overall, load on servers much less, much higher success rate, faster
  event production
  \item Gauginos, Stau, H recoil, Sleptons productions should be done by mid
  April
  \item Triple H and ttbar delayed
\end{itemize}

Consider also cleaning old unused data: please provide me a list of ProdIDs to
keep. No rush.
\end{frame}

\end{document}