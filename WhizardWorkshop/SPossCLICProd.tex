%\documentclass[handout]{beamer}
\documentclass{beamer}
%\usepackage{beamerarticle}
\usepackage{tikz}
\usetikzlibrary{arrows,shapes}

\author{S. Poss}
\title{Use of Whizard for the mass production of CLIC CDR volume 2}
%\subtitle{Update}
\institute[CERN]
{
  CERN
}
\date{22/11/11}
\mode<presentation>
{
   \setbeamertemplate{navigation symbols}{}
   \setbeamertemplate{footline}[frame number] 
}

\AtBeginSection[]
{
\begin{frame}<beamer>
\frametitle{Outline}
\tableofcontents[currentsection,currentsubsection]
\end{frame}
}


\begin{document}
\begin{frame}
\titlepage
\end{frame}
\begin{frame}
\frametitle{Outline}
\tableofcontents
% You might wish to add the option [pausesections]
\end{frame}
\tikzstyle{decision} = [diamond, draw, text badly centered, node distance=2.8cm]
\tikzstyle{block} = [rectangle, draw, text centered, rounded corners, text width=1.9cm, node distance=2.2cm]
\tikzstyle{autoblock} = [rectangle, draw, text centered, rounded corners, node distance=2.2cm]
\tikzstyle{line} = [draw, -triangle 90]
\tikzstyle{dline} = [draw, dashed, -triangle 90]
%\tikzstyle{cloud} = [draw, ellipse,fill=red!20, node distance=3cm,minimum height=2em]
\section{Introduction}

\begin{frame}
\frametitle{Introduction}

\end{frame}
\section{Motivation for the CDR}
\begin{frame}
\frametitle{Motivation for the CDR}
\end{frame}

\section{Samples considered}
\begin{frame}
\frametitle{Physics channels studied}

\end{frame}
\section{Tools used and developped}

\begin{frame}
\frametitle{Tools developped}

\end{frame}
\section{Conclusion}
\begin{frame}
\frametitle{Conclusion}

\end{frame}
\end{document}