%\documentclass[handout]{beamer}
\documentclass{beamer}
%\usepackage{beamerarticle}
\usepackage{tikz}
\usetikzlibrary{arrows,shapes}
\usepackage{heppennames}
\usepackage{hepnicenames}

\newcommand{\whizard}{WHIZARD\xspace}

\author{S. Poss for the LCD group}
\title{Use of \whizard for the mass production of CLIC CDR volume 2}
%\subtitle{Update}
\institute[CERN]
{
  CERN
}
\date{November 22nd, 2011}
\mode<presentation>
{
   \setbeamertemplate{navigation symbols}{}
   \setbeamertemplate{footline}[frame number] 
}

\AtBeginSection[]
{
\begin{frame}<beamer>
\frametitle{Outline}
\tableofcontents[currentsection,currentsubsection]
\end{frame}
}


\begin{document}
\begin{frame}
\titlepage
\end{frame}
\begin{frame}
\frametitle{Outline}
\tableofcontents
% You might wish to add the option [pausesections]
\end{frame}
\tikzstyle{decision} = [diamond, draw, text badly centered, node distance=2.8cm]
\tikzstyle{block} = [rectangle, draw, text centered, rounded corners, text width=1.9cm, node distance=2.2cm]
\tikzstyle{autoblock} = [rectangle, draw, text centered, rounded corners, node distance=2.2cm]
\tikzstyle{line} = [draw, -triangle 90]
\tikzstyle{dline} = [draw, dashed, -triangle 90]
%\tikzstyle{cloud} = [draw, ellipse,fill=red!20, node distance=3cm,minimum height=2em]
\section{Introduction}

\begin{frame}
\frametitle{Introduction}
CLIC studies use ILC as baseline: \alert{use common tools}. For generation,
linear collider common generator tools group: use \whizard1.95 as proved to be
good for ILC LOIs.\\
~\\
CLIC detector studies to be presented in the Conceptual Design Report (CDR):
study 6 physics channels to assess detector performance.\\
~\\
This presentation:
\begin{itemize}
  \item Channels studied
  \item Tools developed for mass production
  \item Limitations
\end{itemize}
\end{frame}

\section{Samples considered}
\begin{frame}
\frametitle{Physics channels studied}
6 benchmark channels to assess detector performance:
\begin{itemize}
\item $\Pep\Pem \to \Ph \Pgne \Pagne, \Ph \to \mu^+\mu^-, \Ph \to
\Pbottom\APbottom$,
\item  $\Pep\Pem \to \PHp \PHm$, $\Pep\Pem \to \PHz \PA$, 
\item $\Pep\Pem \to \PSq_R \PaSq_R$, 
\item $\Pep\Pem \to \PSl \PaSl \,(\ell = \Pe,\Pgm)$, 
\item $\Pep\Pem \to \PSgxpm_i \PSgxmp_j,\, \Pep\Pem \to \PSgxz_i \PSgxz_j$,
\item  $\Pep\Pem \to \Pqt \Paqt$ (500~GeV).
\end{itemize}
2 different SUSY models, 2 different energies (beam spectra), many background
samples, millions of events needed
\end{frame}

\begin{frame}
\frametitle{Examples of results}
\begin{center}
\includegraphics[width=5cm]{xsec_vs_Hmass_sqrts3000}
\includegraphics[width=5cm]{ee_h_mumu_mass_mh120GeV}\\
\includegraphics[width=5cm]{MassPlot2D}
\includegraphics[width=5cm]{205_H1LPADC4}
\end{center}

\end{frame}

\section{Mods needed and Tools developped}
\begin{frame}
\frametitle{Modifications needed}
ILC LOIs used \whizard.\\
~\\
Several tweaks implemented (T.~Barklow, M.~Berggren, A.~Miyamoto):
\begin{itemize}
  \item Installation tools
  \item Specific support for beam spectra: chose between ILC, CLIC, etc.
  \item TAUOLA support
  \item Stdhep output
  \item 6 fermions final state handling (for ILC LOIs): $\Ptop\APtop$
  reconstruction
  \item Color flow (for DBD studies)
\end{itemize}
\end{frame}

\begin{frame}
\frametitle{Production framework}
CLIC detector studies uses the {\color{blue} GRID} for mass production: need to
run many applications in heterogeneous context: \alert{ILCDIRAC}
\begin{itemize}
  \item Based on DIRAC: grid solution developed initially for the LHCb
  experiment. {\color{blue}PYTHON} based.
  \item Comes with production system: define task and leave the system create
  and monitor the jobs.
\end{itemize}
~\\
Needed to interface \whizard in ILCDIRAC:
\begin{itemize}
  \item Run configuration: ILCDIRAC configuration is PYTHON
  \item Process selection: Make sure all relevant files (e.g. LesHouches files)
  are available
\end{itemize}
\end{frame}

\begin{frame}
\frametitle{Running \whizard in ILCDIRAC}
Problem: how to tell \whizard what to do when not called directly?
\begin{itemize}
  \item Need to  configure a workflow object with set of instructions
  \item Call \whizard after setting whizard.in accordingly
\end{itemize}
~\\
Problem: How to prevent users from running non existent channels?
\begin{itemize}
  \item Store the content of the whizard.prc in a file that can be used at
  submission time by DIRAC
  \item Keep relation between \whizard version and channels (SM vs SUSY)
\end{itemize}
~\\
Problem: How to reduce configuration issues?
\begin{itemize}
  \item Catch all errors before the job is submitted,
  \item New functionality from last week: all \whizard options are wrapped in a
  XML file, also holds default values and types
\end{itemize}
\end{frame}

\begin{frame}
\frametitle{Running \whizard in the production}
\begin{itemize}
  \item Framework is global: generation, simulation and reconstruction of events
  done using same tool (ILCDIRAC)
  \item Account for simulation/reconstruction CPU time constrains on the GRID
  (24 hours max on average)
  \item Account for optimal file sizes: Storage Element access is the most
  problematic
\end{itemize}
$\Rightarrow$ \alert{need to generate $10-1000$ events per job} depending on the
channel.
\end{frame}
\section{Limitations met}
\begin{frame}
\frametitle{Limitations met}
\begin{itemize}
  \item Impossible to generate specific final states with width: $\Ptop\APtop$,
  WW, ZZ. {\color{blue}Used PYTHIA standalone} for those. For Higgs processes
  assume width to be 0.\\
  ~\\
  \item Would have been useful to be able to {\color{blue}kill diagrams
  explicitely}\\
  ~\\
  \item Generator level cut interface not easily generalized: had to
  {\color{blue}add extra filtering programs} to run after \whizard.\\
  ~\\
  \item Process selection for people not compiling \whizard: Adding a
  {\color{blue}new process requires compiling a \whizard executable}, not easy
  from scratch
\end{itemize}
\end{frame}
\section{The future}
\begin{frame}
\frametitle{Using \whizard 2.0}
Using analysis framework of \whizard2:
\begin{center}
\includegraphics[width=7cm]{model3}
\end{center}
Includes CLIC 3TeV luminosity spectrum.
\end{frame}
\section{Conclusion}
\begin{frame}
\frametitle{Conclusion}
For the CDR:
\begin{itemize}
  \item Very complete software
  \item Accurate cross sections
  \item Efficient event generation
  \item Lightweight to run on the GRID once compiled
  \item Compilation is tricky
  \item Some channels could not be done ($\Ptop\APtop$, WW, ZZ)
\end{itemize}
For \whizard 2.0:
\begin{itemize}
  \item Process selection
  \item Run configuration
  \item 8 fermions final states (ILC DBD studies)?
\end{itemize}
\end{frame}

\appendix
\begin{frame} 
\frametitle{Backup slides}
\end{frame}
\begin{frame}
\frametitle{Requests for the future}
\begin{itemize}
  \item Kill diagrams
  \item Do not depend on compiler at run time
  \item Interface to other languages (e.g. XML)
\end{itemize}
\end{frame}
\end{document}