\documentclass[14pt]{beamer}


\usepackage{color}
\usepackage{tikz}


\mode<presentation>
{
\usetheme{AlpesLasers}
\setbeamercovered{transparent}
  %\setbeamertemplate{footline}[frame number] 
  %\setbeamertemplate{navigation symbols}{ 
  %\hskip 0.3cm
  %\insertframenumber / \inserttotalframenumber  % <<< frame #
  %\insertpagenumber / \insertpresentationendpage % <<< page #
%} 
}

% font definitions, try \usepackage{ae} instead of the following
% three lines if you don't like this look
\usepackage{listings}
\lstloadlanguages{python}

\usepackage{mathptmx}
\usepackage[scaled=.90]{helvet}
\usepackage{courier}
\usepackage[T1]{fontenc}
\usepackage[english]{babel}
\usepackage[latin1]{inputenc}
\title{Building an efficient simulation framework}
\subtitle{A little overview}
\author{St\'ephane Poss}
\date{\today}
% This is only inserted into the PDF information catalog. Can be left
% out.
\subject{PYTHON}

\lstdefinestyle{custompy}{
  belowcaptionskip=1\baselineskip,
  breaklines=true,
  xleftmargin=\parindent,
  language=python,
  showstringspaces=false,
  basicstyle=\footnotesize\ttfamily,
  keywordstyle=\bfseries\color{green!40!black},
  commentstyle=\itshape\color{purple!40!black},
  identifierstyle=\color{blue},
  stringstyle=\color{orange},
}
\lstdefinestyle{customsh}{
  belowcaptionskip=1\baselineskip,
  breaklines=true,
  xleftmargin=\parindent,
  language=bash,
  showstringspaces=false,
  basicstyle=\footnotesize\ttfamily,
  keywordstyle=\bfseries\color{green!40!black},
  commentstyle=\itshape\color{purple!40!black},
  identifierstyle=\color{blue},
  stringstyle=\color{orange},
}
\lstdefinestyle{customcpp}{
  belowcaptionskip=1\baselineskip,
  breaklines=true,
  xleftmargin=\parindent,
  language=C++,
  showstringspaces=false,
  basicstyle=\footnotesize\ttfamily,
  keywordstyle=\bfseries\color{green!40!black},
  commentstyle=\itshape\color{purple!40!black},
  identifierstyle=\color{blue},
  stringstyle=\color{orange},
}

\begin{document}
\begin{frame}[plain]
\titlepage
\end{frame}

\begin{frame}
\tableofcontents
\end{frame}

\section{The need}
\begin{frame}
\frametitle{Why do we need this?}
\begin{itemize}
\item Statistical trust in results: significance of one point?
\pause
\item Optimize: 'best' set of parameters' values
\pause
\item Investigate phase space: impact of correlations
\begin{itemize}
\item How will a structure behave at different temperature/higher current/different geometry?\\ 
\pause
\alert{$\rightarrow$ Predict behavior}
\end{itemize}
\pause
\item Numerical validation when no analytical model available
\end{itemize}
\end{frame}

\section{The goal}
\begin{frame}
\frametitle{What we want to achieve}
\begin{block}{}
Analyse many phase space points quickly: \\try 1000s of combinaisons
\end{block}
\begin{block}{}
Access results efficiently\\
Ensure reproducibility
\end{block}
\end{frame}

\section{Utilities}
\begin{frame}
\frametitle{Intermission}
\centering
\includegraphics[width=\textwidth]{python-logo-master-v3-TM.png}
\end{frame}

\begin{frame}
\frametitle{Why PYTHON?}
\begin{itemize}
\item Modern Object Oriented language
\item Large user base and lots of libraries
\item Suitable for task definition
\item Flexible
\item Easy to write/read
\item No compilation needed
\end{itemize}
See other presentations.
\end{frame}

\begin{frame}
\frametitle{Version control}
\begin{block}{Why?}
\begin{itemize}
\item Compare different version of an analysis
\item Share with others
\item Traceability
\end{itemize}
\end{block}
\end{frame}
\begin{frame}
\frametitle{Version control}
\begin{block}{How?}
\begin{itemize}
\item[SVN] Centralized: need to get all changes from master server every time
\item[\alert{Git}] Distributed: all history local, can have many parallel copies
\end{itemize}
\includegraphics[width=0.5\textwidth]{svn-name-banner.jpg}$\quad$
\includegraphics[width=0.3\textwidth]{Git-Logo-2Color.png}
\end{block}
\end{frame}

\begin{frame}
\frametitle{Version control}
\centering
\alert{Version control is essential!}
\end{frame}

\section{Computing}
\begin{frame}
\frametitle{Computing paradigms}
\begin{itemize}
\item[Parallelize] Process complex computations using all cores of a PC or many cores of a supercomputer. Uses Threading/Multiprocess. Tasks exchange data while running.
\item[Distribute] Send multiple times the same task with different inputs to different unrelated computers.
\end{itemize}
MPI is in between.
\end{frame}

\section{Program wrapping}
\begin{frame}
\frametitle{Execute a program indirectly}
Why?
\begin{itemize}
\item Distribute computations
\begin{itemize}
\item Configure individual runs
\item Control the program's behavior: error catching/recovery
\end{itemize}
\item Simplify the interface: expose only the essential components
\end{itemize}
$\rightarrow$ wrap it!
\end{frame}

\begin{frame}
\frametitle{Wrapping}
\begin{itemize}
\item Produce the steering files / command line arguments
\item Control them
\item Call the program (PYTHON's subprocess)
\item Check the output
\end{itemize}
$\rightarrow$ Make sure no intervention is required
\end{frame}


\section{Collecting results}
\begin{frame}
\frametitle{Data collection}
Results should be available
\begin{itemize}
\item quickly
\item in a logical way
\end{itemize}
\end{frame}

\subsection{Text files}
\begin{frame}
\frametitle{Text files}
\begin{itemize}
\item Large files
\item Complex to parse when data > 3d
\item Loss of precision due to truncation
\item Usualy has header documentation 
\end{itemize}
Example: CSV
\end{frame}

\subsection{Binary files}
\begin{frame}
\frametitle{Binary file storage}
\begin{itemize}
\item Full precision
\item Hard to parse
\item Separate documentation
\item Can be compressed
\end{itemize}
Examples: HDF5, netcdf
\end{frame}

\subsection{Database}
\begin{frame}
\frametitle{Database systems}
\begin{itemize}
\item Use relations between data (RDMS)
\item Many systems exist: MySQL, PostgresSQL, ORACLE, SQLite, etc.
\item Use logical queries to get data
\item Interface complex: description of data model, access to server, etc.
\begin{itemize}
\item SQLAlchemy
\end{itemize}
\end{itemize}
Other system: SciDB
\end{frame}


\section{User interfaces}
\begin{frame}
\frametitle{User Interfaces categories}
\begin{block}{One user}
Simple command line: parameter validation done by user coding
\end{block}
\begin{block}{Many users}
Parameter validation\\
Options presented
\end{block}
\end{frame}

\begin{frame}
\frametitle{UI Types}
\begin{block}{Command Line Interface}
Simple/straightforward, clumsy, ugly, portable
\end{block}
\begin{block}{Graphical User Interfaces}
Harder to code, good looking, not portable as requires client specific code
\end{block}
\begin{block}{Web GUI}
Mix of both, requires server+client, very portable
\end{block}
\end{frame}

\begin{frame}
\frametitle{Intermediate summary}
We have now:
\begin{itemize}
\item A program that can be executed indirectly
\item Reproducibility
\item A means to control the I/O
\item A way to store the results
\item User convenience
\end{itemize}
What we need:
\begin{itemize}
\item Run many times similar things
\end{itemize}

\centering
\includegraphics[width=0.5\textwidth]{Dirac_logo_RGB.png}

\end{frame}


\end{document}